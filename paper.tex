\documentclass[10pt, conference, letterpaper]{IEEEtran}

\usepackage{booktabs}
\usepackage{multirow}
\usepackage{listings} 
\usepackage{graphicx}
\usepackage{enumitem}
\usepackage{subcaption}
\usepackage{tabularx}
\usepackage{float}
\usepackage{amsthm}
\usepackage{amsmath}
\usepackage{algorithm}
\usepackage{setspace}
\usepackage[noend]{algpseudocode}
\usepackage{algorithmicx}
\usepackage{makecell}
\usepackage{multicol}
\usepackage{listings}
\usepackage{verbatim}
\usepackage{fancyvrb}
\usepackage{lineno}
\usepackage[square,numbers]{natbib}
\usepackage{lipsum}
\usepackage{syntax}
\usepackage{cleveref}
\usepackage{layouts}
\usepackage{tablefootnote}
\usepackage{footnote}
\usepackage{tikz}
\usepackage{bbding}
\usepackage{pifont}
\usepackage{wasysym}
\usepackage{amssymb}
\usepackage{url}
\usepackage{xspace}
\usepackage{diagbox}
\usepackage[flushleft]{threeparttable}
\usepackage{alltt,xcolor}
\usepackage[justification=centering]{caption}
\usepackage{csquotes,filecontents}

\usepackage{color}

\usepackage{titlesec}

%\titleformat*{\section}{\normalsize\bfseries}
\titlespacing{\section}{0pt}{4pt}{4pt}
\titlespacing{\subsection}{0pt}{4pt}{4pt}
\titlespacing{\subsubsection}{0pt}{4pt}{4pt}

\definecolor{OliveGreen}{cmyk}{0.64,0,0.95,0.40}
\definecolor{ao}{rgb}{0.0, 0.5, 0.0}
\definecolor{asparagus}{rgb}{0.53, 0.66, 0.42}
\definecolor{applegreen}{rgb}{0.55, 0.71, 0.0}
\definecolor{aogreen}{rgb}{0.0, 0.5, 0.0}
\definecolor{columbiablue}{rgb}{0.61, 0.87, 1.0}
\definecolor{cornellred}{rgb}{0.7, 0.11, 0.11}
\definecolor{cornflowerblue}{rgb}{0.39, 0.58, 0.93}
\definecolor{denim}{rgb}{0.08, 0.38, 0.74}

\colorlet{BtfulGreen}{black!30!green!70!}
\colorlet{BtfulOrange}{white!5!orange!95!}
\colorlet{BtfulGray}{white!50!gray!50!}

\algnewcommand\INPUT{\item[\textbf{Input:}]}%
\algnewcommand\OUTPUT{\item[\textbf{Output:}]}%
\newcommand{\algorithmicbreak}{\textbf{break}}
\newcommand{\BREAK}{\State \algorithmicbreak}

\newcommand{\specialcell}[2][c]{%
  \begin{tabular}[#1]{@{}c@{}}#2\end{tabular}}

\newcommand{\specialcellleft}[2][c]{%
  \begin{tabular}[#1]{@{}l@{}}#2\end{tabular}}

\newcommand{\specialcellright}[2][c]{%
  \begin{tabular}[#1]{@{}r@{}}#2\end{tabular}}

\setlength{\paperheight}{11in}
\setlength{\paperwidth}{8.5in}
\tolerance=1000

\setlist{nolistsep}
\setlength{\abovecaptionskip}{0.8pt}
\setlength{\floatsep}{5pt}
\setlength{\textfloatsep}{1pt}

\setlength{\belowcaptionskip}{0.5pt}
%\setlength{\skip\footins}{0pt}

\newcommand{\para}[1]{\noindent\textbf{#1}}
\newcommand{\cut}[1]{}
\newcommand{\etal}{{\em et al.}}

\newcommand{\sysname}{ArchMon\xspace}
\newcommand{\gurobi}{ArchMon-OPT\xspace}
\newcommand{\true}{\centering$\checkmark$}
\newcommand{\false}{\centering$\times$}

\newcommand{\vA}{\mathbf{A}}
\newcommand{\vx}{\vec{x}}
\newcommand{\vy}{\vec{y}}
\newcommand{\vz}{\vec{z}}
\newcommand{\qun}[1]{\textcolor{red}{(QUN: #1)}}
\newcommand{\todo}[1]{\textcolor{red}{(TODO: #1)}}
\newcommand{\red}[1]{{\color{red}#1}}

\newtheorem{theorem}{Theorem}
\newtheorem{corollary}{Corollary}
\newtheorem{definition}{Definition}
\newtheorem{lemma}{Lemma}
\newtheorem{proposition}{Proposition}

\hyphenation{ArchMon}

\setitemize{noitemsep,topsep=0em,parsep=0em,partopsep=3pt}

\DeclareRobustCommand*{\IEEEauthorrefmark}[1]{%
    \raisebox{0pt}[0pt][0pt]{\textsuperscript{\footnotesize\ensuremath{#1}}}}

\newcommand*\circled[1]{\tikz[baseline=(char.base)]{
            \node[shape=circle,draw,inner sep=0.7pt] (char) {#1};}}

% \IEEEoverridecommandlockouts\IEEEpubid{\makebox[\columnwidth]{ Haifeng Zhou is the corresponding author. \hfill}\hspace{\columnsep}\makebox[\columnwidth]{ }}

\begin{document}

\title{\makebox[\linewidth]{\parbox{\dimexpr\textwidth+1cm\relax}{\centering Rearchitecting Network Measurement with Accurate and Resource-Efficient Sketch-INT Co-Design}}\vspace{-0.1in}}
% \author{
%      \IEEEauthorblockN{Xiang Chen\IEEEauthorrefmark{1,}\IEEEauthorrefmark{2}, Xi Sun\IEEEauthorrefmark{1}, Wenbin Zhang\IEEEauthorrefmark{1}, Xin Yao\IEEEauthorrefmark{1}, Zizheng Wang\IEEEauthorrefmark{1}, Hongyan Liu\IEEEauthorrefmark{1},\\ Qun Huang\IEEEauthorrefmark{3}, Gaoning Pan\IEEEauthorrefmark{4}, Xuan Liu\IEEEauthorrefmark{5,}\IEEEauthorrefmark{6}, Haifeng Zhou\IEEEauthorrefmark{1}, Chunming Wu\IEEEauthorrefmark{2,}\IEEEauthorrefmark{1}}
%      \IEEEauthorblockA{\IEEEauthorrefmark{1}Zhejiang University \IEEEauthorrefmark{2}Quan Cheng Laboratory \IEEEauthorrefmark{3}Peking University \IEEEauthorrefmark{4}Hangzhou Dianzi University}  \IEEEauthorrefmark{5}Yangzhou University \IEEEauthorrefmark{6}Southeast University\vspace{-0.1in}}


% make the title area
\maketitle

\pagenumbering{gobble}

\fontdimen2\font=0.1ex
\begin{abstract}
Network measurement is now a fundamental building block that enables network management to detect and handle real-time events (e.g., rejecting malicious flows). However, state-of-the-art techniques present a trade-off between high accuracy and resource efficiency: sketches enjoy low resource consumption when measuring large flows. But they sacrifice accuracy for small flows. Also, in-band network telemetry (INT) accurately measures each flow at the cost of significant resources.
In this paper, we present \sysname, a framework that utilizes sketches and INT to keep track of large and small flows, respectively, simultaneously achieving high accuracy and resource efficiency. More precisely, unlike existing studies, our co-design is driven by two theoretical optimizations: (1) \sysname adopts the near-optimal Lagrangian relaxation to solve the NP-hard selection of measurement points to deploy sketches and INT with unknown routing information. (2) \sysname estimates the worst-case rates of collecting measurement data. Its estimates support reinforcement learning to select paths without congestions.  
Experiments on 64$\times$400\,Gbps Tofino switches indicate that compared to existing techniques, \sysname provides higher accuracy (e.g., $5\times$ F1 scores) and lower resource consumption (e.g., reducing bandwidth overhead by 70\%). 
\end{abstract}

%Sketches naturally produce accurate results for large flows with resource efficiency. Next, \sysname uses INT to transfer the data of small flows to the control plane. As small flows only have a few packets, it limits INT's resource consumption. 

\fontdimen2\font=3pt
\section{Introduction}

Network measurement has been a fundamental component in modern networks. For example, it enables traffic monitoring and engineering in data center networks \cite{li2016flowradar,woodruff2019measuring,zhang2017high} and wide-area networks \cite{huang2017sketchvisor,liu2016one,caida}, and supports network latency monitoring in distributed large language model training scenarios \cite{qian2024alibaba}. More precisely, it measures real-time traffic statistics such as per-flow packet counts in data plane switches and periodically collects these measurement data to the control plane. The control plane offers data to network management applications for identifying the events of their interest and making corresponding decisions. 

However, existing measurement techniques face the dilemma of achieving high accuracy and resource efficiency simultaneously. More precisely, the literature has proposed two categories of measurement techniques, i.e., sketches and in-band network telemetry (INT). For sketches \cite{li2016flowradar,yang2018elastic,huang2017sketchvisor,huang2018sketchlearn,liu2016one,huang2021toward,liu2019nitrosketch,zhang2021cocosketch,namkung2022sketchlib}, they achieve accurate measurement for large flows via compact data structures. While sketches are resource-efficient, they exhibit low accuracy when measuring small flows due to the short of switch memory. For INT-based approaches \cite{sheng2021deltaint,ben2020pint,handigol2014know,pan2019int,zhu2015packet}, they record statistics within the INT headers of each packet and collect these headers at the egress of networks. Although their per-packet monitoring achieves full accuracy, it imposes high pressure on network bandwidth and control plane resources to handle large volumes of INT data. 

Although recent studies attempt to break the above dilemma, they come at additional cost. Some of them propose sophisticated sketches such as elastic sketches \cite{yang2018elastic} and pyramid sketches \cite{yang2017pyramid} that distinguish the measurement of small flows from large flows to mitigate the errors originated from hashing collisions. However, they involve complex data structures and operations, which hinder their hardware implementation, and also require time-consuming parameter tuning. Also, a lot of optimizations aim at reducing INT overheads via strategies like path planning \cite{pan2019int} or probabilistic updating \cite{ben2020pint}. Nevertheless, given the large volume of large flows, massive packets collectively force INT to generate a high-speed stream of INT headers, which brings non-trivial pressure to the control plane. 

Hereby, we issue the following question: how can we achieve accurate and resource-efficient measurement towards both large and short flows? Our answer is \sysname, which is an accurate and resource-efficient measurement framework. Our core idea originates from our observation: sketches and INT techniques are \emph{complementary} to each other. More precisely, sketches trade their accuracy of measuring small flows for resource efficiency at scale, while INT techniques trade their resource efficiency for full accuracy of measuring every flow. Hence, their advantages and limitations lead to the possible match: for traffic in modern networks, which is considered skewed (i.e., most packets come from a few large flows) \cite{roy2015inside,huang2021toward,caida,benson2010network,yang2018elastic}, \sysname leverages sketches to measure large flows while using INT techniques to measure small flows, maximizing both accuracy and efficiency.

\para{Contributions}. However, co-designing sketches and INT techniques within the same system faces the fundamental challenges of \emph{where}, \emph{how}, and \emph{how efficiently} to deploy these techniques. In response, we design \sysname to bridge the accuracy-efficiency gap by systematically integrating sketches and INT. Our major contributions are summarized as follows.  

\begin{itemize}[leftmargin=*]
%
    \item[1] \emph{Near-optimal measurement point selection}. Given a network and a set of orientation-destination (OD) pairs, each of which represents the ingress and egress of a specific flow, \sysname chooses enough switches to place sketches and INT. It aims at maximizing the coverage of measured flows and minimizing the distance between the switches where flows are measured and the control plane where traffic statistics are collected and analyzed. Next, it formulates such an optimization as a multi-objective facility location problem and leverages Lagrangian relaxation to obtain near-optimal decisions. 
%
    \item[2] \emph{Congestion-free measurement data transmission}. At runtime, according to its decisions, \sysname runs sketches and INT on selected switches. Sketches periodically report their data of large flows to the control plane, while INT utilizes every packet of small flows to transfer data. To prevent these techniques from producing a high-speed stream of measurement data that causes congestion, \sysname profiles the maximum possible rate of sending measurement data from every switch to the control plane. According to profiling results, it decides which paths to transfer data with a provably high probability of avoiding congestion. 
%
    \item[3] \emph{Accurate and resource-efficient measurement data reduction}. Directly emitting a large volume of measurement data to the control plane in a short time can easily saturate control plane resources, while existing approaches \cite{chen2021mtp} are inefficient since they fail to change their static allocation to deal with traffic dynamics. In response, \sysname incorporates a suite of data reduction techniques, i.e., merging similar data, only sending data delta, and data caching, into its data collection to reduce the amount of measurement data by orders of magnitude. 
%
    \item[4] \emph{Real-world implementation and evaluation}. We have implemented \sysname on a testbed that comprises $64\times 400$\,Gbps Tofino2 switches \cite{tofino2}. Our testbed experiments indicate that \sysname outperforms existing standalone sketches or INT techniques with significant accuracy improvement (e.g., $3\times$ F1 score) and orders-of-magnitude lower overheads. 
%
\end{itemize}


\section{Background and Motivation}\label{background}

\begin{figure}
    \centering
    \includegraphics[width=\linewidth]{pics/DesignSpace.png}
    \caption{\sysname explicitly separates the processing of large flows (handled by sketches) and small flows (handled by INT) based on traffic skewness, making it accurate and efficient.}
    \label{DesignSpace}
\end{figure}

\subsection{Our Scope: Network Measurement and Applications}

We first illustrate our scope in this paper. We follow existing studies \cite{namkung2022sketchlib,anup2022hetero,liu2016one} to consider a general architecture of network measurement comprising two planes. In the data plane, switches execute sketches and INT techniques to collect traffic statistics to a cluster of servers that collectively act as the control plane. The control plane runs network management applications that query statistics to identify network events and make decisions. We define relevant terms without loss of generality \cite{liu2024disco,wu2025lemon}:

\begin{itemize}[leftmargin=*]
%
    \item Each packet can be viewed as a tuple of header fields (e.g., IP) and metadata (e.g., packet size and location traversed).
%
    \item Each flow groups the packets that share the same set of header fields and metadata, e.g., the same source IP address. 
%
    \item Traffic statistics, or per-flow statistics, correspond to the flow attributes of interest, e.g., per-flow packet counts. 
%
\end{itemize}

% \noindent With the above terms, we classify network management applications into three classes based on their differences on queries.

% \begin{itemize}[leftmargin=*]
% %
%     \item Volumetric applications, including heavy hitter detection \cite{huang2017sketchvisor}, superspreader detection \cite{tang2019mv}, and DDoS flow detection \cite{liu2021jaqen}, query the size of each flow (e.g., packet or byte counts). 
% %
%     \item Aggregated applications, e.g., traffic cardinality and entropy estimation \cite{liu2016one}, aggregate statistics to summarize all flows. 
% %
%     \item Troubleshooting applications, e.g., path and latency monitoring \cite{ben2020pint,sheng2021deltaint}, query per-flow metadata (e.g., timestamps).
% %
% \end{itemize}

%\noindent Also, since measurements may suffer from errors, the literature introduces several application-level accuracy metrics to evaluate the quality of measurements: the precision, \emph{tp}/(\emph{tp}+\emph{fp}), the recall, \emph{tp}/(\emph{tp}+\emph{fn}), and the F1 score, 2\emph{tp}/(2\emph{tp}+\emph{fp}+\emph{fn}), where \emph{tp}, \emph{fp}, and \emph{fn} denote the number of true positives, false positives, and false negatives, respectively. 

\subsection{Sketches and Limitations}\label{sketches}

Sketches are the approximate algorithms that provide near-optimal accuracy guarantees to the measurement of large flows. Their resource consumption is below a limited budget, making them compatible with resource-constrained switches. Numerous studies \cite{li2016flowradar,yang2018elastic,huang2017sketchvisor,huang2018sketchlearn,liu2016one,huang2021toward,liu2019nitrosketch,zhang2021cocosketch,namkung2022sketchlib} have shown that sketches provide better accuracy-resource tradeoffs for large flows when compared with sampling-based techniques.
% such as NetFlow \cite{netflow}. 
%Note that our work is not to propose new sketches. Instead, we focus on leveraging existing sketches to enable better measurements. 

\para{Limitation 1 (L1):} \emph{Sketches can hardly provide high accuracy for small flows}. Sketches can only use a few memory in their data structures due to switch resource constraints. As a result, the data of large flows and small flows easily collide, resulting in measurement errors. In particular, the measurement of small flows is extremely sensitive to such errors because these flows have very few packets and may be missed entirely or their small counts may be overestimated due to hash collisions. 

% figure a: from nze-sketch background, x = sketches, y = ratio
% figure b: x = the ratio of switches activating INT from 20\% to 100\% (100% = 152 switches), y = number of INT headers, each switch handles 1 Tbps traffic => each packet generates an INT header

We validate our claim with testbed experiments. We consider five sketches and employ them for the application of per-flow packet counting, i.e., the count-min sketch (CM) \cite{cormode2005s}, the count sketch (CS) \cite{charikar2004finding}, the elastic sketch (ES) \cite{yang2018elastic}, SketchLearn (SL) \cite{huang2018sketchlearn}, and UnivMon (UM) \cite{liu2016one}. We consider the de-facto standard traffic traces, CAIDA \cite{caida}, and partition them into two-second intervals, each containing 100\,K flows. Then for each sketch, we allocate 10\,MB to its data structures because this configuration approximates the maximum memory capacity per switch \cite{gupta2018sonata}. 
We measure the ratio of flows with $<10$\% errors in each sketch. Due to space limitations, we report numerical results: all ratios are lower than 50\% because most small flows suffer from non-trivial errors due to memory shortage. 

%If we allocate sufficient memory to each sketch, the accuracy will be improved. However, this is infeasible due to switch constraints. 

Some recent studies have been proposed to tackle the above issue. For example, NZE-sketch \cite{huang2021toward} leverages the theories of compressive sensing to design sketches to approximately record traffic data in switches, and recover original data in the control plane. Moreover, learning-based sketches such as TalentSketch \cite{li2024learning} leverage machine learning models to identify which flows suffer from low measurement accuracy and update sketch data structures in subsequent measurements. However, they exhibit similar issues. First, they need specific sketch design to enforce their optimizations, losing the generality of supporting different types of sketches or network management applications. Second, their recovery or updating strategies require O(10$^2$) seconds, making them unsuitable for a lot of scenarios in which measurements should be made within a few milliseconds. For example, DDoS detection addresses sub-millisecond detection time \cite{liu2021jaqen}. 

\subsection{In-Band Network Telemetry (INT) and Limitations}\label{int}

The core idea of INT is simple yet effective. For each packet, INT orders every switch that the packet traverses to piggyback the information required for measurement onto the packet's INT header. At the egress switch, where packets leave the network, their INT headers are extracted and emitted to the control plane for further analysis. Thus, INT preserves full accuracy for every flow, especially for small flows. 

\para{Limitation 2 (L2):} \emph{INT can bring significant network resource overheads}. Since INT corresponds to per-packet monitoring, it inevitably brings significant overheads to both network bandwidth (for transferring INT headers) and control plane resources (for processing INT headers) given massive packet number. In theory, according to the INT protocol \cite{int}, each switch will add a 12-byte INT header to each packet while the total number of INT headers equals the number of hops traversed by the packet. Given that modern networks transfer Tbps-level traffic, the total number of packets per second is O(10$^9$), corresponding to the number of INT headers to be emitted simultaneously. Hence, although 12\,bytes seem to be small, the accumulation of O(10$^9$) INT headers becomes significant and unacceptable. 

%Figure~1 highlights the impact of number of switches in the network on INT overheads. In detail, we consider the topology of a production network from one of the largest national Internet service providers. This network has 152 switches and 225 links. We vary the ratio of switches activating INT from 20\% to 100\%. We quantify the overheads in terms of number of INT headers to be collected with respect to real-world statistics (e.g., how many packets that a switch processes per second) collected from production. Our results show that even with 20\% switches, INT still introduces non-trivial overheads in terms of bandwidth and computational resources in the control plane. 

To date, many optimizations have been proposed to optimize INT overheads. First, some of them rely on sampling and only invoke INT for sampled packets \cite{tang2019sel,suh2020flexible,kim2018selective,ben2020pint}. Nevertheless, sampling hurts accuracy, especially for small flows with a few packets. Also, the control plane requires a long time for result convergence. Next, INT-path \cite{pan2019int} sends probes to determine the minimum set of non-overlapped paths that INT must measure. But its per-packet monitoring still suffers from high overheads. 
%
Second, some approaches propose to simplify INT information on each packet \cite{ben2020pint,zhao2021lightguardian,sheng2021deltaint}. PINT \cite{ben2020pint} reduces INT metadata via sampling-based global hashing and distributed encoding. But it inherits the issues of sampling. 

\subsection{Limitations of Hybrid Sketch+INT Solutions}\label{hybrid}

Two recent techniques provide hybrid solutions that combine sketches and INT. First, SketchINT \cite{yang2023sketchint} activates INT at data plane switches while leveraging sketches at the control plane to aggregate INT data and simplify subsequent analysis. However, due to its base of INT, it inherits \textbf{L2}. Also, its use of sketches trades the accuracy of small flows for control plane efficiency, leading to \textbf{L1}. Second, LightGuardian \cite{zhao2021lightguardian} uses small sketches to encode INT data in each packet in a compact manner. Hence, it addresses \textbf{L2} because it only requires a few packets to deliver a small amount of sketch data via INT. But given the nature of sketches, it inevitably loses accuracy for small flows, i.e., \textbf{L1}. 

%\para{Takeaways}. Standalone sketches, INT, and strawman sketch-INT hybrid solutions, fail to address \textbf{L1} and \textbf{L2} simultaneously.  


%To maximize resource efficiency, measurement data should be aggressively reduced. But existing techniques \cite{sheng2021deltaint,chen2021mtp,liu2022escala} only consider one type of measurement data and naturally fail to handle the fundamental heterogeneity between sketch data (i.e., sketch counter arrays) and INT data (i.e., INT headers). Also, reducing data must preserve integrity, i.e., 

%For example, many approaches \cite{sheng2021deltaint,chen2021mtp,liu2022escala} only consider one type of measurement data. Thus, they naturally lack a uniform way to reduce heterogeneous data. 

%Traditional static data reduction techniques fail to handle skewed and time-varying traffic patterns, leading to either insufficient compression (wasting bandwidth) or over-aggregation (missing critical details). This is because measurement data scale depends on the flow size: large flows tolerate summarization, while small flows require precision. For example, sketch data alone can be compressed by merging similar values, but INT data may reduce a 1MB sketch update to 100KB during stable periods but fails during traffic shifts (e.g., DDoS onset), where 95\% of counters change simultaneously. Similarly, caching frequent items risks missing ephemeral but critical microburst indicators. This forces inefficient tradeoffs: aggressive reduction distorts small-flow accuracy, while conservative approaches flood control planes during volatility.

%TowerSketch \cite{}


\section{Overview of \sysname}\label{overview}

\para{Goals}. We aim to achieve two goals of network measurement.

\begin{itemize}[leftmargin=*]
%
    \item \textbf{G1: High accuracy}, i.hello e., measuring large flows and small flows accurately (addressing \textbf{L1}, \S\ref{sketches}). 
%
    \item \textbf{G2: Resource efficiency}, i.e., avoiding high resource overheads in measurement data collection (addressing \textbf{L2}, \S\ref{int}). 
%
\end{itemize}

\para{Key idea}. According to our analysis, we observe that sketches and INT complement each other. More precisely, in Table~I, we summarize the advantages and limitations of sketches and INT: Sketches exhibit both high accuracy and resource efficiency for large flows, but they fall short for small flows; INT can provide high accuracy for both large and small flows, but they come at the cost of high resource consumption. 

In this context, we conclude the opportunities of combining sketches and INT: we can use sketches to measure large flows while using INT to measure small flows. Doing so can achieve both high accuracy and resource efficiency. In detail, traffic in modern networks is skewed \cite{roy2015inside,huang2021toward,caida,benson2010network,yang2018elastic}. So most packets originate from large flows, which are accurately and efficiently measured by sketches. For other packets from small flows, they are measured by INT, ensuring their full accuracy. Also, their number is small, limiting INT's resource consumption. Hence, both \textbf{G1} and \textbf{G2} can be achieved. 

\para{Challenges}.
However, while it is technically sound, leveraging these opportunities faces three system-level challenges.

\begin{itemize}[leftmargin=*]
%
    \item \textbf{C1: Measurement point selection with incomplete knowledge}. Traffic routing details are usually unknown in advance. As such, selecting measurement points (i.e., the switches that execute sketches and INT) is non-trivial. First, guaranteeing that every flow traverses at least one point when routing paths are unknown is challenging. This uncertainty may result in over-provisioning of measurement points, increasing deployment overheads. Second, measurement points should be close to the control plane. Otherwise, measurement data collection may suffer from high latency, delaying network management. Thus, without precise path knowledge, achieving both high coverage and low collection latency is NP-hard. 
%
    \item \textbf{C2: General data reduction for heterogeneous measurement data}. For resource efficiency, measurement data should be minimized to save network bandwidth and control plane resources. However, such reduction faces the inherent heterogeneity between sketch data (e.g., structured counter arrays) and INT headers (e.g., unstructured hop-by-hop metadata). Specifically, reducing sketch data needs to preserve counter relationships over time, while reducing INT data should avoid modifying original per-switch information. Such differences impede the design of reducing measurement data. Without careful analysis, simply collaborating existing data reduction techniques would be ineffective and even hurt data integrity. 
%
    \item \textbf{C3: Congestion-free measurement data collection under bursts}. Measurement data from sketches and INT generates data streams that may saturate network paths connecting measurement points with the control plane. Thus, this challenge stems from the need to prevent the streams of measurement data from causing congestion during collection. For example, when thousands of incoming flows simultaneously activate INT at the switch, the resulting O(10$^2$)\,Gbps data traffic can overwhelm 100-Gbps paths in milliseconds. Thus, avoiding congestion requires complicated modeling of worst-case collection rates to prevent collisions with normal traffic. 
    %Profiling ``safe'' collection rates alone is insufficient because traffic dynamics cause rates to vary unpredictably. 
%
\end{itemize}

\para{Design}. In response, we design \sysname, a framework that co-designs sketches and INT to collectively achieve high accuracy (\textbf{G1}) and resource efficiency (\textbf{G2}). To address the above challenges, \sysname embraces three-pronged system mechanisms. 

\begin{itemize}[leftmargin=*]
%
    \item \emph{Near-optimal measurement point selection}. Given a network and a set of orientation-destination (OD) pairs, each of which represents the ingress and egress of a specific flow, \sysname chooses enough switches to place sketches and INT. It aims at maximizing the coverage of measured flows and minimizing the distance between the switches where flows are measured and the control plane where traffic statistics are collected and analyzed. Next, it formulates such an optimization as a multi-objective facility location problem and leverages Lagrangian relaxation to obtain near-optimal decisions (addressing \textbf{C1}). 
%
    \item \emph{Resource-efficient measurement data reduction}. At runtime, sketches report their data that summaries large flow statistics to the control plane while INT keeps track of small flows. However, when all measurement data, including sketch and INT data, is collected to the control plane, their large volume can easily saturate control plane resources in a short time. In response, we analyze the characteristics of various types of measurement data. With analysis results, we design \sysname to adopt a suite of techniques, e.g., merging similar data and only sending data delta, in data collection to safely minimize data while preserving integrity (addressing \textbf{C2}). 
%
    \item \emph{Congestion-free measurement data collection}. Switches collectively report high-speed streams of measurement data to the control plane via normal network paths. Therefore, these streams may be collided with normal traffic, causing network congestion and significant data loss. To prevent congestions, \sysname automatically profiles the maximum possible rate of sending measurement data from every switch to the control plane. According to profiling results, it decides which paths to transfer data while avoiding congestion (addressing \textbf{C3}). 
%
\end{itemize}

\para{Architecture}. Figure~2 presents the architecture of \sysname, which employs a five-step workflow. 

\begin{itemize}[leftmargin=*]
%
    \item[1] Administrators submit the sketches and INT techniques they plan to use to \sysname. As a general framework, \sysname supports arbitrary types of sketches and INT techniques. 
%
    \item[2] Without precise information, \sysname solves the selection of measurement points via a Lagrangian relaxation algorithm that yields near-optimal decisions within polynomial time.
    %\sysname formulates the selection of measurement points that deploy sketches and INT to cover as many flows as possible into a multi-objective facility location problem \cite{karatas2018iterative} with coverage constraints. Thereafter, it solves this problem via a Lagrangian relaxation algorithm that yields near-optimal decisions within polynomial time. 
%
    \item[3] \sysname performs sketches and INT atop selected switches. Each flow will be first measured by sketches. If it is identified as a large flow, its data is approximately recorded by sketches. Otherwise, its data is piggybacked on its headers via INT. 
%
    \item[4] Before emitting measurement data, i.e., periodically collecting sketch data or sending INT headers, \sysname minimizes data by using the techniques that best match each data type. 
    %\sysname minimizes data to reduce resource consumption. It first characterizes the differences between sketch data and INT data. After that, it uses corresponding data reduction techniques that best match each data type to minimize data. 
%
    \item[5] \sysname selects network paths to transfer measurement data with the goal of avoiding congestion with normal traffic. The control plane receives data, which is input to applications for subsequent network management. 
    %For each type of data emitted by each switch, \sysname estimates the worst-case rate of sending such data. According to 
%
\end{itemize}


%\para{Step-by-step example}. 



\section{Design of \sysname}\label{design}
% \input{design}
% \input{heuristic}
% \input{placer}
% \input{evaluation1}
% % %\input{related-work}
% \section{Conclusion}

We propose \sysname, a framework that co-designs sketches and INT to measure large and small flows respectively for accurate and resource-efficient network measurement. To enable such a co-design, 
\sysname solves the NP-hard measurement point selection with the near-optimal Lagrangian relaxation. It offers congestion-free data collection with worst-case rate estimations of sending measurement data and RL-based dynamic path selection. We implement
\sysname on 64$\times$400\,Gbps Tofino switches. Our testbed experiments indicate that \sysname offers remarkable accuracy improvement (e.g., 5.1$\times$ higher F1 scores) while achieving low resource consumption.




\clearpage
%\balance

%{\small
%\bibliographystyle{abbrv}
%\bibliography{paper}
%}

%\section*{Acknowledgement}

%We thank our reviewers for their insightful and constructive comments. 
%\para{Acknowledgement}. We thank reviewers for their comments. This work is supported by ``Pioneer'' and ``Leading Goose'' R\&D Program of Zhejiang (2024C01066), Quan Cheng Laboratory (QCLZD202304), Natural Science Foundation of China (61902362, 62172007), the Provincial Key R\&D Program of Zhejiang (2021C01032), the Yangzhou Science and Technology Plan Project (YZ2023200), Self-Developing Experimental Instrument and Equipment Project of Yangzhou University (zzyq2023zy06), the Fundamental Research Funds for the Central Universities (Zhejiang University NGICS Platform), and the Major Science and Technology Infrastructure Project of Zhejiang Lab (Large-scale experimental device for information security of new generation industrial control system).

%This work is supported by the National Key R\&D Program of China (2019YFB1802600), the National Natural Science Foundation of China (61902362, and 62172007), the Joint Funds of the National Natural Science Foundation of China (U20A20179), the Fundamental Research Funds for the Central Universities (Zhejiang University NGICS Platform), and the Major Science and Technology Infrastructure Project of Zhijiang Laboratory (Large-scale experimental device for information security of new generation industrial control system).

%the Key R\&D Program of Zhejiang Province (2021C01036), 

%\clearpage
\renewcommand\refname{Reference}
\bibliographystyle{IEEEtran}
{\footnotesize
\bibliography{paper}}

% that's all folks
\end{document}
