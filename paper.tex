\documentclass[10pt, conference, letterpaper]{IEEEtran}

\usepackage{booktabs}
\usepackage{multirow}
\usepackage{listings} 
\usepackage{graphicx}
\usepackage{enumitem}
\usepackage{subcaption}
\usepackage{tabularx}
\usepackage{float}
\usepackage{amsthm}
\usepackage{amsmath}
\usepackage{algorithm}
\usepackage{setspace}
\usepackage[noend]{algpseudocode}
\usepackage{algorithmicx}
\usepackage{makecell}
\usepackage{multicol}
\usepackage{listings}
\usepackage{verbatim}
\usepackage{fancyvrb}
\usepackage{lineno}
\usepackage[square,numbers]{natbib}
\usepackage{lipsum}
\usepackage{syntax}
\usepackage{cleveref}
\usepackage{layouts}
\usepackage{tablefootnote}
\usepackage{footnote}
\usepackage{tikz}
\usepackage{bbding}
\usepackage{pifont}
\usepackage{wasysym}
\usepackage{amssymb}
\usepackage{url}
\usepackage{xspace}
\usepackage{diagbox}
\usepackage[flushleft]{threeparttable}
\usepackage{alltt,xcolor}
\usepackage[justification=centering]{caption}
\usepackage{csquotes,filecontents}

\usepackage{color}

\usepackage{titlesec}

%\titleformat*{\section}{\normalsize\bfseries}
\titlespacing{\section}{0pt}{4pt}{4pt}
\titlespacing{\subsection}{0pt}{4pt}{4pt}
\titlespacing{\subsubsection}{0pt}{4pt}{4pt}

\definecolor{OliveGreen}{cmyk}{0.64,0,0.95,0.40}
\definecolor{ao}{rgb}{0.0, 0.5, 0.0}
\definecolor{asparagus}{rgb}{0.53, 0.66, 0.42}
\definecolor{applegreen}{rgb}{0.55, 0.71, 0.0}
\definecolor{aogreen}{rgb}{0.0, 0.5, 0.0}
\definecolor{columbiablue}{rgb}{0.61, 0.87, 1.0}
\definecolor{cornellred}{rgb}{0.7, 0.11, 0.11}
\definecolor{cornflowerblue}{rgb}{0.39, 0.58, 0.93}
\definecolor{denim}{rgb}{0.08, 0.38, 0.74}

\colorlet{BtfulGreen}{black!30!green!70!}
\colorlet{BtfulOrange}{white!5!orange!95!}
\colorlet{BtfulGray}{white!50!gray!50!}

\algnewcommand\INPUT{\item[\textbf{Input:}]}%
\algnewcommand\OUTPUT{\item[\textbf{Output:}]}%
\newcommand{\algorithmicbreak}{\textbf{break}}
\newcommand{\BREAK}{\State \algorithmicbreak}

\newcommand{\specialcell}[2][c]{%
  \begin{tabular}[#1]{@{}c@{}}#2\end{tabular}}

\newcommand{\specialcellleft}[2][c]{%
  \begin{tabular}[#1]{@{}l@{}}#2\end{tabular}}

\newcommand{\specialcellright}[2][c]{%
  \begin{tabular}[#1]{@{}r@{}}#2\end{tabular}}

\setlength{\paperheight}{11in}
\setlength{\paperwidth}{8.5in}
\tolerance=1000

\setlist{nolistsep}
\setlength{\abovecaptionskip}{0.8pt}
\setlength{\floatsep}{5pt}
\setlength{\textfloatsep}{1pt}

\setlength{\belowcaptionskip}{0.5pt}
%\setlength{\skip\footins}{0pt}

\newcommand{\para}[1]{\noindent\textbf{#1}}
\newcommand{\cut}[1]{}
\newcommand{\etal}{{\em et al.}}

\newcommand{\sysname}{MonPlan\xspace}
\newcommand{\gurobi}{MonPlan-OPT\xspace}
\newcommand{\true}{\centering$\checkmark$}
\newcommand{\false}{\centering$\times$}

\newcommand{\vA}{\mathbf{A}}
\newcommand{\vx}{\vec{x}}
\newcommand{\vy}{\vec{y}}
\newcommand{\vz}{\vec{z}}
\newcommand{\qun}[1]{\textcolor{red}{(QUN: #1)}}
\newcommand{\todo}[1]{\textcolor{red}{(TODO: #1)}}
\newcommand{\red}[1]{{\color{red}#1}}

\newtheorem{theorem}{Theorem}
\newtheorem{corollary}{Corollary}
\newtheorem{definition}{Definition}
\newtheorem{lemma}{Lemma}
\newtheorem{proposition}{Proposition}

\hyphenation{MonPlan}

\setitemize{noitemsep,topsep=0em,parsep=0em,partopsep=3pt}

\DeclareRobustCommand*{\IEEEauthorrefmark}[1]{%
    \raisebox{0pt}[0pt][0pt]{\textsuperscript{\footnotesize\ensuremath{#1}}}}

\newcommand*\circled[1]{\tikz[baseline=(char.base)]{
            \node[shape=circle,draw,inner sep=0.7pt] (char) {#1};}}

% \IEEEoverridecommandlockouts\IEEEpubid{\makebox[\columnwidth]{ Haifeng Zhou is the corresponding author. \hfill}\hspace{\columnsep}\makebox[\columnwidth]{ }}

\begin{document}

\title{\makebox[\linewidth]{\parbox{\dimexpr\textwidth+1cm\relax}{\centering Optimizing Network Measurement with Accurate and Resource-Efficient Sketch-INT Co-Design}}\vspace{-0.1in}}
% \author{
%      \IEEEauthorblockN{Xiang Chen\IEEEauthorrefmark{1,}\IEEEauthorrefmark{2}, Xi Sun\IEEEauthorrefmark{1}, Wenbin Zhang\IEEEauthorrefmark{1}, Xin Yao\IEEEauthorrefmark{1}, Zizheng Wang\IEEEauthorrefmark{1}, Hongyan Liu\IEEEauthorrefmark{1},\\ Qun Huang\IEEEauthorrefmark{3}, Gaoning Pan\IEEEauthorrefmark{4}, Xuan Liu\IEEEauthorrefmark{5,}\IEEEauthorrefmark{6}, Haifeng Zhou\IEEEauthorrefmark{1}, Chunming Wu\IEEEauthorrefmark{2,}\IEEEauthorrefmark{1}}
%      \IEEEauthorblockA{\IEEEauthorrefmark{1}Zhejiang University \IEEEauthorrefmark{2}Quan Cheng Laboratory \IEEEauthorrefmark{3}Peking University \IEEEauthorrefmark{4}Hangzhou Dianzi University}  \IEEEauthorrefmark{5}Yangzhou University \IEEEauthorrefmark{6}Southeast University\vspace{-0.1in}}


% make the title area
\maketitle

\pagenumbering{gobble}

\fontdimen2\font=0.1ex
\begin{abstract}
Network measurement is now a fundamental building block that enables network management to detect and handle real-time events (e.g., rejecting malicious flows). However, state-of-the-art techniques present a trade-off between high accuracy and resource efficiency: sketches enjoy low resource consumption when measuring large flows. But they sacrifice accuracy for small flows. Also, in-band network telemetry (INT) accurately measures each flow at the cost of significant resources.
In this paper, we present \sysname, a framework that utilizes sketches and INT to keep track of large and small flows, respectively, simultaneously achieving high accuracy and resource efficiency. More precisely, unlike existing studies, our co-design is driven by two theoretical optimizations: (1) \sysname adopts the near-optimal Lagrangian relaxation to solve the NP-hard selection of measurement points to deploy sketches and INT with unknown routing information. (2) \sysname estimates the worst-case rates of collecting measurement data. Its estimates support reinforcement learning to select paths without congestions.  
Experiments on 64$\times$400\,Gbps Tofino switches indicate that compared to existing techniques, \sysname provides higher accuracy (e.g., $5\times$ F1 scores) and lower resource consumption (e.g., reducing bandwidth overhead by 70\%). 
\end{abstract}

%Sketches naturally produce accurate results for large flows with resource efficiency. Next, \sysname uses INT to transfer the data of small flows to the control plane. As small flows only have a few packets, it limits INT's resource consumption. 

\fontdimen2\font=3pt
\section{Introduction}

Network measurement has been a fundamental component in modern networks. For example, it enables traffic monitoring and engineering in data center networks \cite{li2016flowradar,woodruff2019measuring,zhang2017high} and wide-area networks \cite{huang2017sketchvisor,liu2016one,caida}, and supports network latency monitoring in distributed large language model training scenarios \cite{qian2024alibaba}. More precisely, it measures real-time traffic statistics such as per-flow packet counts in data plane switches and periodically collects these measurement data to the control plane. The control plane offers data to network management applications for identifying the events of their interest and making corresponding decisions. 

However, existing measurement techniques face the dilemma of achieving high accuracy and resource efficiency simultaneously. More precisely, the literature has proposed two categories of measurement techniques, i.e., sketches and in-band network telemetry (INT). For sketches \cite{li2016flowradar,yang2018elastic,huang2017sketchvisor,huang2018sketchlearn,liu2016one,huang2021toward,liu2019nitrosketch,zhang2021cocosketch,namkung2022sketchlib}, they achieve accurate measurement for large flows via compact data structures. While sketches are resource-efficient, they exhibit low accuracy when measuring small flows due to the short of switch memory. For INT-based approaches \cite{sheng2021deltaint,ben2020pint,handigol2014know,pan2019int,zhu2015packet}, they record statistics within the INT headers of each packet and collect these headers at the egress of networks. Although their per-packet monitoring achieves full accuracy, it imposes high pressure on network bandwidth and control plane resources to handle large volumes of INT data. 

Although recent studies attempt to break the above dilemma, they come at additional cost. Some of them propose sophisticated sketches such as elastic sketches \cite{yang2018elastic} and pyramid sketches \cite{yang2017pyramid} that distinguish the measurement of small flows from large flows to mitigate the errors originated from hashing collisions. However, they involve complex data structures and operations, which hinder their hardware implementation, and also require time-consuming parameter tuning. Also, a lot of optimizations aim at reducing INT overheads via strategies like path planning \cite{pan2019int} or probabilistic updating \cite{ben2020pint}. Nevertheless, given the large volume of large flows, massive packets collectively force INT to generate a high-speed stream of INT headers, which brings non-trivial pressure to the control plane. 

While some hybrid systems combine both INT and sketches \cite{yang2023sketchint,zhao2021lightguardian}, they still fall short. Typical examples are SketchINT \cite{yang2023sketchint} that builds sketches at the control plane based on collected INT data to rapidly reply flow queries while LightGuardian \cite{zhao2021lightguardian} that inserts sketch data into INT headers to achieve resource-efficient measurement data collection. But SketchINT inherits the cost of INT while both SketchINT and LightGuardian suffer from low accuracy for small flows due to the nature of sketches. 

Hereby, we issue the following question: how can we achieve accurate and resource-efficient measurement towards both large and small flows? Our answer is \sysname, which is an accurate and resource-efficient measurement framework. Our core idea originates from our observation: sketches and INT techniques are \emph{complementary} to each other. More precisely, sketches trade their accuracy of measuring small flows for resource efficiency at scale, while INT techniques trade their resource efficiency for full accuracy of measuring every flow. Hence, their advantages and limitations lead to the possible match: for traffic in modern networks, which is considered skewed (i.e., most packets come from a few large flows) \cite{roy2015inside,huang2021toward,caida,benson2010network,yang2018elastic}, \sysname leverages sketches to measure large flows while using INT techniques to measure small flows, maximizing both accuracy and efficiency.

While this concept is intuitive, combining sketches with INT faces two optimization challenges: \textbf{C1}: where to install sketches and INT to measure as many flows as possible when routing information is unknown, and \textbf{C2}: how to prevent measurement data collection from congesting network paths and affecting normal traffic still remain hard optimization problems. Ignoring \textbf{C1} leads to high measurement errors due to the miss of massive flows while neglecting \textbf{C2} results in severe measurement data loss during traffic bursts, e.g., existing hybrid systems \cite{yang2023sketchint,zhao2021lightguardian} that incorporate both sketches and INT ignore these challenges and suffer from accuracy and efficiency loss (\S\ref{eval}). To address these challenges, \sysname offers two optimizations, including measurement point selection that yields near-optimal decisions under unknown routing information, and congestion-free data collection that profiles worst-case estimates to determine loss-free decisions. Such optimizations enable \sysname to achieve both high accuracy and resource efficiency in practice. 

%(2) resource-efficient measurement data reduction that efficiently and safely reduces diverse types of data, 

%how to efficiently reduce the large amount of measurement data to improve resource efficiency while keeping data integrity, 

\para{Contributions}. 
%However, co-designing sketches and INT techniques within the same system faces the fundamental challenges of \emph{where}, \emph{how}, and \emph{how efficiently} to deploy these techniques. 
%we design \sysname to bridge the accuracy-efficiency gap by systematically integrating sketches and INT. 
We summarize our contributions as follows.  

\begin{itemize}[leftmargin=*]
%
    \item We present an in-depth analysis towards the advantages and limitations of existing measurement techniques. Our analysis presents that sketches provide accurate and resource-efficient measurements for large flows but fall short for small flows, while INT accurately monitors all flows but suffers from high resource overheads (\S\ref{background}). 
%
    \item According to analysis results, combining sketches and INT presents an opportunity of realizing both high accuracy and resource efficiency simultaneously. We exploit this opportunity by proposing \sysname, a framework that uses sketches to measure large flows while invoking INT for small flows, enabling accurate and resource-efficient measurements (\S\ref{overview}). 
%
    \item However, co-designing sketches and INT faces the challenges of \emph{where} to deploy them and \emph{which paths} to collect their data. \sysname provides two optimizations, i.e., near-optimal measurement point selection, and congestion-free measurement data collection to handle these challenges (\S\ref{selection}-\S\ref{collection}). 
    % resource-efficient measurement data reduction, as well as
%
    \item We implement \sysname on our testbed that comprises $64\times 400$\,Gbps Tofino2 switches \cite{tofino2}. Testbed experiments indicate that \sysname outperforms existing standalone sketches or INT techniques with accuracy improvement (e.g., $3\times$ F1 scores) and orders-of-magnitude lower overheads. 
%
\end{itemize}


\section{Background and Motivation}\label{background}

\subsection{Our Scope: Network Measurement and Applications}

We first illustrate our scope in this paper. We follow existing studies \cite{namkung2022sketchlib,anup2022hetero,liu2016one} to consider a general architecture of network measurement comprising two planes. In the data plane, switches execute sketches and INT techniques to collect traffic statistics to a cluster of servers that collectively act as the control plane. The control plane runs network management applications that query statistics to identify network events and make decisions. We define relevant terms without loss of generality as follows.

\begin{itemize}[leftmargin=*]
%
    \item Each packet can be viewed as a tuple of header fields (e.g., IP) and metadata (e.g., packet size and location traversed).
%
    \item Each flow groups the packets that share the same set of header fields and metadata, e.g., the same source IP address. 
%
    \item Traffic statistics, or per-flow statistics, correspond to the flow attributes of interest, e.g., per-flow packet counts. 
%
\end{itemize}

\noindent With the above terms, we classify network management applications into three classes based on their differences on queries.

\begin{itemize}[leftmargin=*]
%
    \item Volumetric applications, including heavy hitter detection \cite{huang2017sketchvisor}, superspreader detection \cite{tang2019mv}, and DDoS flow detection \cite{liu2021jaqen}, query the size of each flow (e.g., packet or byte counts). 
%
    \item Aggregated applications, e.g., traffic cardinality and entropy estimation \cite{liu2016one}, aggregate statistics to summarize all flows. 
%
    \item Troubleshooting applications, e.g., path and latency monitoring \cite{ben2020pint,sheng2021deltaint}, query per-flow metadata (e.g., timestamps).
%
\end{itemize}

%\noindent Also, since measurements may suffer from errors, the literature introduces several application-level accuracy metrics to evaluate the quality of measurements: the precision, \emph{tp}/(\emph{tp}+\emph{fp}), the recall, \emph{tp}/(\emph{tp}+\emph{fn}), and the F1 score, 2\emph{tp}/(2\emph{tp}+\emph{fp}+\emph{fn}), where \emph{tp}, \emph{fp}, and \emph{fn} denote the number of true positives, false positives, and false negatives, respectively. 

\subsection{Sketches and Limitations}\label{sketches}

Sketches are the approximate algorithms that provide near-optimal accuracy guarantees to the measurement of large flows. They keep their resource consumption below a limited resource budget. So they are compatible with resource-constrained data plane switches. Prior studies \cite{li2016flowradar,yang2018elastic,huang2017sketchvisor,huang2018sketchlearn,liu2016one,huang2021toward,liu2019nitrosketch,zhang2021cocosketch,namkung2022sketchlib} have shown that sketches provide better accuracy-resource tradeoffs for large flows when compared with traditional sampling-based techniques such as NetFlow \cite{netflow}. Note that our work is not to propose new sketches. Instead, we focus on leveraging existing sketches to enable better measurements. 

\para{Limitation 1 (L1):} \emph{Sketches can hardly provide high accuracy for small flows}. Sketches can only use a few memory in their data structures due to switch resource constraints. As a result, the data of large flows and small flows easily collide, resulting in measurement errors. In particular, the measurement of small flows is extremely sensitive to such errors because these flows have very few packets and may be missed entirely or their small counts may be overestimated due to hash collisions. 

% figure a: from nze-sketch background, x = sketches, y = ratio
% figure b: x = the ratio of switches activating INT from 20\% to 100\% (100% = 152 switches), y = number of INT headers, each switch handles 1 Tbps traffic => each packet generates an INT header

We validate our claim with testbed experiments. We consider five sketches and employ them for the application of per-flow packet counting, i.e., the count-min sketch (CM) \cite{cormode2005s}, the count sketch (CS) \cite{charikar2004finding}, the elastic sketch (ES) \cite{yang2018elastic}, SketchLearn (SL) \cite{huang2018sketchlearn}, and UnivMon (UM) \cite{liu2016one}. We consider the de-facto standard traffic traces, CAIDA \cite{caida}, and partition them into two-second intervals, each containing 100\,K flows. Then for each sketch, we allocate 10\,MB to its data structures because this configuration approximates the maximum memory capacity per switch \cite{gupta2018sonata}. In Figure~1, we present the ratio of flows, which measurement errors are below 10\%, in each sketch. All ratios are lower than 50\%. The reason is that most small flows suffer from non-trivial errors due to memory shortage. If we allocate sufficient memory to each sketch (i.e., ``Ideal''), the accuracy would be improved. However, this is infeasible in practice due to switch constraints. 

Some recent studies have been proposed to tackle the above issue. For example, NZE-sketch \cite{huang2021toward} leverages the theories of compressive sensing to design sketches to approximately record traffic data in switches, and recover original data in the control plane. Moreover, learning-based sketches such as TalentSketch \cite{li2024learning} leverage machine learning models to identify which flows suffer from low measurement accuracy and update sketch data structures in subsequent measurements. However, they exhibit similar issues. First, they need specific sketch design to enforce their optimizations, losing the generality of supporting different types of sketches or network management applications. Second, their recovery or updating strategies require O(10$^2$) seconds, making them unsuitable for a lot of scenarios in which measurements should be made within a few milliseconds. For example, DDoS detection addresses sub-millisecond detection time \cite{liu2021jaqen}. 

\subsection{In-Band Network Telemetry (INT) and Limitations}\label{int}

The core idea of INT is simple yet effective. For each packet, INT orders every switch that the packet traverses to piggyback the information required for measurement onto the packet's INT header. At the egress switch, where packets leave the network, their INT headers are extracted and emitted to the control plane for further analysis. Thus, INT preserves full accuracy for every flow, especially for small flows. 

\para{Limitation 2 (L2):} \emph{INT can bring significant network resource overheads}. Since INT corresponds to per-packet monitoring, it inevitably brings significant overheads to both network bandwidth (for transferring INT headers) and control plane resources (for processing INT headers) given massive packet number. In theory, according to the INT protocol \cite{int}, each switch will add a 12-byte INT header to each packet while the total number of INT headers equals the number of hops traversed by the packet. Given that modern networks transfer Tbps-level traffic, the total number of packets per second is O(10$^9$), corresponding to the number of INT headers to be emitted simultaneously. Hence, although 12\,bytes seem to be small, the accumulation of O(10$^9$) INT headers becomes significant and unacceptable. 

Figure~1 highlights the impact of number of switches in the network on INT overheads. In detail, we consider the topology of a production network from one of the largest national Internet service providers. This network has 152 switches and 225 links. We vary the ratio of switches activating INT from 20\% to 100\%. We quantify the overheads in terms of number of INT headers to be collected with respect to real-world statistics (e.g., how many packets that a switch processes per second) collected from production. Our results show that even with 20\% switches, INT still introduces non-trivial overheads in terms of bandwidth and computational resources in the control plane. 

To date, many optimizations have been proposed to optimize INT overheads. First, some of them rely on sampling and only invoke INT for sampled packets \cite{tang2019sel,suh2020flexible,kim2018selective,ben2020pint}. Nevertheless, sampling hurts accuracy, especially for small flows with a few packets. Also, the control plane requires a long time for result convergence. Next, INT-path \cite{pan2019int} sends probes to determine the minimum set of non-overlapped paths that INT must measure. But its per-packet monitoring still suffers from high overheads. 
%
Second, some approaches propose to simplify INT information on each packet \cite{ben2020pint,zhao2021lightguardian,sheng2021deltaint}. PINT \cite{ben2020pint} reduces INT metadata via sampling-based global hashing and distributed encoding. But it inherits the issues of sampling. 

\subsection{Limitations of Hybrid Sketch+INT Solutions}\label{hybrid}

Two recent techniques provide hybrid solutions that combine sketches and INT. First, SketchINT \cite{yang2023sketchint} activates INT at data plane switches while leveraging sketches at the control plane to aggregate INT data and simplify subsequent analysis. However, due to its base of INT, it inherits \textbf{L2}. Also, its use of sketches trades the accuracy of small flows for control plane efficiency, leading to \textbf{L1}. Second, LightGuardian \cite{zhao2021lightguardian} uses small sketches to encode INT data in each packet in a compact manner. Hence, it addresses \textbf{L2} because it only requires a few packets to deliver a small amount of sketch data via INT. But given the nature of sketches, it inevitably loses accuracy for small flows, i.e., \textbf{L1}. 

\para{Takeaways}. Standalone sketches, INT, and strawman sketch-INT hybrid solutions, fail to address \textbf{L1} and \textbf{L2} simultaneously.  


%To maximize resource efficiency, measurement data should be aggressively reduced. But existing techniques \cite{sheng2021deltaint,chen2021mtp,liu2022escala} only consider one type of measurement data and naturally fail to handle the fundamental heterogeneity between sketch data (i.e., sketch counter arrays) and INT data (i.e., INT headers). Also, reducing data must preserve integrity, i.e., 

%For example, many approaches \cite{sheng2021deltaint,chen2021mtp,liu2022escala} only consider one type of measurement data. Thus, they naturally lack a uniform way to reduce heterogeneous data. 

%Traditional static data reduction techniques fail to handle skewed and time-varying traffic patterns, leading to either insufficient compression (wasting bandwidth) or over-aggregation (missing critical details). This is because measurement data scale depends on the flow size: large flows tolerate summarization, while small flows require precision. For example, sketch data alone can be compressed by merging similar values, but INT data may reduce a 1MB sketch update to 100KB during stable periods but fails during traffic shifts (e.g., DDoS onset), where 95\% of counters change simultaneously. Similarly, caching frequent items risks missing ephemeral but critical microburst indicators. This forces inefficient tradeoffs: aggressive reduction distorts small-flow accuracy, while conservative approaches flood control planes during volatility.

%TowerSketch \cite{}


\section{Overview of \sysname}\label{overview}

\para{Goals}. We aim to achieve two goals of network measurement.

\begin{itemize}[leftmargin=*]
%
    \item \textbf{G1: High accuracy}, i.e., measuring large flows and small flows accurately (addressing \textbf{L1}, \S\ref{sketches}). 
%
    \item \textbf{G2: Resource efficiency}, i.e., avoiding high resource overheads in measurement data collection (addressing \textbf{L2}, \S\ref{int}). 
%
\end{itemize}

\para{Key idea}. According to our analysis, we observe that sketches and INT complement each other. More precisely, in Table~I, we summarize the advantages and limitations of sketches and INT: Sketches exhibit both high accuracy and resource efficiency for large flows, but they fall short for small flows; INT can provide high accuracy for both large and small flows, but they come at the cost of high resource consumption. 

In this context, we conclude the opportunities of combining sketches and INT: we can use sketches to measure large flows while using INT to measure small flows. Doing so can achieve both high accuracy and resource efficiency. In detail, traffic in modern networks is skewed \cite{roy2015inside,huang2021toward,caida,benson2010network,yang2018elastic}. So most packets originate from large flows, which are accurately and efficiently measured by sketches. For other packets from small flows, they are measured by INT, ensuring their full accuracy. Also, their number is small, limiting INT's resource consumption. Hence, both \textbf{G1} and \textbf{G2} can be achieved. 

\para{Challenges}.
However, while it is technically sound, leveraging these opportunities faces two optimization challenges.

\begin{itemize}[leftmargin=*]
%
    \item \textbf{C1: Measurement point selection with incomplete knowledge}. Traffic routing details are usually unknown in advance. As such, selecting measurement points (i.e., the switches that execute sketches and INT) is non-trivial. First, guaranteeing that every flow traverses at least one point when routing paths are unknown is challenging. This uncertainty may result in over-provisioning of measurement points, increasing deployment overheads. Second, measurement points should be close to the control plane. Otherwise, measurement data collection may suffer from high latency, delaying network management. Thus, without precise path knowledge, achieving both high coverage and low collection latency is NP-hard. 
%
%    \item \textbf{C2: General data reduction for heterogeneous measurement data}. For resource efficiency, measurement data should be minimized to save network bandwidth and control plane resources. However, such reduction faces the inherent heterogeneity between sketch data (e.g., structured counter arrays) and INT headers (e.g., unstructured hop-by-hop metadata). Specifically, reducing sketch data needs to preserve counter relationships over time, while reducing INT data should avoid modifying original per-switch information. Such differences impede the design of reducing measurement data. Without careful analysis, simply collaborating existing data reduction techniques would be ineffective and even hurt data integrity. 
%
    \item \textbf{C2: Congestion-free measurement data collection under bursts}. Measurement data from sketches and INT generates data streams that may saturate network paths connecting measurement points with the control plane. Thus, this challenge stems from the need to prevent the streams of measurement data from causing congestion during collection. For example, when thousands of incoming flows simultaneously activate INT at the switch, the resulting O(10$^2$)\,Gbps data traffic can overwhelm 100-Gbps paths in milliseconds. Thus, avoiding congestion requires complicated modeling of worst-case collection rates to prevent collisions with normal traffic. 
    %Profiling ``safe'' collection rates alone is insufficient because traffic dynamics cause rates to vary unpredictably. 
%
\end{itemize}

\para{Design}. In response, we design \sysname, a framework that co-designs sketches and INT to collectively achieve high accuracy (\textbf{G1}) and resource efficiency (\textbf{G2}). To address the above challenges, \sysname embraces three-pronged system mechanisms. 

\begin{itemize}[leftmargin=*]
%
    \item \emph{Near-optimal measurement point selection}. Given a network and a set of orientation-destination (OD) pairs, each of which represents the ingress and egress of a specific flow, \sysname chooses enough switches to place sketches and INT. It aims at maximizing the coverage of measured flows and minimizing the distance between the switches where flows are measured and the control plane where traffic statistics are collected and analyzed. Next, it formulates such an optimization as a multi-objective facility location problem and leverages Lagrangian relaxation to obtain near-optimal decisions (addressing \textbf{C1}). 
%
%    \item \emph{Resource-efficient measurement data reduction}. At runtime, sketches report their data that summaries large flow statistics to the control plane while INT keeps track of small flows. However, when all measurement data, including sketch and INT data, is collected to the control plane, their large volume can easily saturate control plane resources in a short time. In response, we analyze the characteristics of various types of measurement data. With analysis results, we design \sysname to adopt a suite of techniques, e.g., merging similar data and only sending data delta, in data collection to safely minimize data while preserving integrity (addressing \textbf{C2}). 
%
    \item \emph{Congestion-free measurement data collection}. Switches collectively report high-speed streams of measurement data to the control plane via normal network paths. Therefore, these streams may be collided with normal traffic, causing network congestion and significant data loss. To prevent congestions, \sysname automatically profiles the maximum possible rate of sending measurement data from every switch to the control plane. According to profiling results, it decides which paths to transfer data while avoiding congestion (addressing \textbf{C2}). 
%
\end{itemize}

\para{Architecture}. Figure~2 presents the architecture of \sysname, which employs a four-step workflow. 

\begin{itemize}[leftmargin=*]
%
    \item[1] Administrators submit the sketches and INT techniques they plan to use to \sysname. As a general framework, \sysname supports arbitrary types of sketches and INT techniques. 
%
    \item[2] Without precise information, \sysname solves the selection of measurement points via a Lagrangian relaxation algorithm that yields near-optimal decisions within polynomial time.
    %\sysname formulates the selection of measurement points that deploy sketches and INT to cover as many flows as possible into a multi-objective facility location problem \cite{karatas2018iterative} with coverage constraints. Thereafter, it solves this problem via a Lagrangian relaxation algorithm that yields near-optimal decisions within polynomial time. 
%
    \item[3] \sysname performs sketches and INT atop selected switches. Each flow will be first measured by sketches. If it is identified as a large flow, its data is approximately recorded by sketches. Otherwise, its data is piggybacked on its headers via INT. 
%
%    \item[4] Before emitting measurement data, i.e., periodically collecting sketch data or sending INT headers, \sysname minimizes data by using the techniques that best match each data type. 
    %\sysname minimizes data to reduce resource consumption. It first characterizes the differences between sketch data and INT data. After that, it uses corresponding data reduction techniques that best match each data type to minimize data. 
%
    \item[4] \sysname selects network paths to transfer measurement data with the goal of avoiding congestion with normal traffic. The control plane receives data, which is input to applications for subsequent network management. 
    %For each type of data emitted by each switch, \sysname estimates the worst-case rate of sending such data. According to 
%
\end{itemize}


%\para{Step-by-step example}. 

\section{Near-Optimal Measurement Point Selection}\label{selection}

\para{Overview.} The measurement point selection problem aims to select programmable switches in the network to deploy sketches and INT. It has two objectives: (1) maximizing flow coverage, i.e., measuring as many flows as possible, and (2) minimizing the distance between the control plane and switches that execute network measurement to enable timely data collection.

\begin{table}[t]
\caption{Notation of major symbols used by this paper.}
\label{tab:notation}
\centering
\resizebox{1.0\linewidth}{!}{
\small
\begin{tabular}{p{0.13\linewidth}p{0.95\linewidth}}
\toprule
\textbf{Symbol} & \textbf{Description} \\
\midrule
$G$ & Network topology with switches $V$ and links $E$ \\
$P \subseteq V$ & Programmable switches supporting sketches and INT \\
$C \subseteq V$ & Control plane nodes \\
$\mathcal{F}$ & Set of flows \\
$o_f, d_f$ & Orientation/destination of flow $f \in \mathcal{F}$ \\
$\mathcal{P}_f$ & Programmable switches on \textit{any shortest path} for flow $f$ \\
$\mathcal{F}_p$ & Flows that \textit{can be measured} at switch $p$ ($\{f \mid p \in \mathcal{P}_f\}$) \\
$\delta(p,c)$ & Distance between switch $p$ and control plane node $c$ \\
$\alpha$ & Coverage-distance tradeoff parameter ($0 \leq \alpha \leq 1$) \\
$\lambda_f$ & Lagrange multiplier for flow $f$'s coverage constraints \\
$L(\boldsymbol{\lambda})$ & Lagrangian dual function \\
$\beta_p$ & Switch penalty = $\sum_{f \in \mathcal{F}_p} \lambda_f$ \\
$\gamma_p$ & Data collection distance cost = $(1-\alpha) \min_{c \in C} \delta(p,c)$ \\
$g_f$ & Subgradient = $\sum_{p \in \mathcal{P}_f} u_p - w_f$ \\
$T$ & Iteration count in subgradient optimization \\
$\pi_\theta$ & Trained RL policy network \\
$\mathcal{P}_{p,c}$ & Selected network path from switch $p$ to control plane node $c$ \\
$\pi_{p,c}$ & Fraction of measurement data traffic from $p$ to $c$ \\
$q_e^t$ & Queue depth (bytes) on link $e$ at time $t$ \\
$\Gamma_p^{\text{total}}$ & Worst-case measurement data rate (bps) at switch $p$ \\
$\Gamma_e^{\text{data}}$ & Normal data plane traffic rate (bps) on link $e$ \\
$\tau_e$ & Safety threshold for congestion avoidance on link $e$ \\
$s_t$ & State vector at time $t$ for RL policy \\
$x_p$ & 0-1 variable = 1 if sketch deployed at switch $p$ \\
$y_p$ & 0-1 variable = 1 if INT deployed at switch $p$ \\
$u_p$ & 0-1 variable = 1 if switch $p$ is selected ($x_p=1 \lor y_p=1$) \\
$z_{p,c}$ & 0-1 variable = 1 if data emitted from $p$ to control plane node $c$ \\
$w_f$ & 0-1 variable = 1 if flow $f$ is covered (i.e., measured) \\
\bottomrule
\end{tabular}}
\end{table}

% \begin{table}[t]
% \caption{Notation of major symbols used by this paper.}
% \label{notation}
% \newcommand{\tabincell}[2]{\begin{tabular}{@{}#1@{}}#2\end{tabular}}
% \centering
% \resizebox{1.0\linewidth}{!}{
% \small
% \begin{tabular}{p{0.13\linewidth}p{0.95\linewidth}}
% \toprule
% \textbf{Symbol} & \textbf{Description} \\
% \midrule
% $G$ & Network $G = (V, E)$: $V$ contains switches; $E$ contains links \\
% $P \subseteq V$ & Programmable switches that enable sketches and INT \\
% $C \subseteq V$ & Control plane nodes \\
% $\mathcal{F}$ & Set of flows \\
% $o_f, d_f$ & Orientation/destination of flow $f \in \mathcal{F}$ \\
% $\mathcal{P}_f$ & Programmable switches on shortest paths for flow $f$ \\
% $\delta(p,c)$ & Distance between switch $p$ and control node $c$ \\
% $\alpha$ & Coverage-distance tradeoff parameter ($0 \leq \alpha \leq 1$) \\
% $x_p$ & 0-1 variable indicating sketch deployment at switch $p$ \\
% $y_p$ & 0-1 variable indicating INT deployment at switch $p$ \\
% $u_p$ & 0-1 variable indicating measurement activation at switch $p$ \\
% $z_{p,c}$ & 0-1 variable indicating data routing from $p$ to control node $c$ \\
% $w_f$ & 0-1 variable indicating flow coverage \\
% \bottomrule
% \end{tabular}}
% \end{table}

\para{Input.} The input comprises:
(1) the network $G = (V, E)$;
(2) programmable switches $P \subseteq V$ supporting sketches and INT;
(3) control plane nodes $C \subseteq V$;
(4) a set $\mathcal{F}$ of flows, where each flow $f \in \mathcal{F}$ only indicates an OD pair $(o_f, d_f)$ \cite{liu2016one,anup2022hetero}, where $o_f$ and $d_f$ are the ingress and egress of $f$, respectively; 
(5) for each flow $f$, the set $\mathcal{P}_f \subseteq P$ of switches on \emph{any shortest path} between $o_f$ and $d_f$;
and (6) the distance metric $\delta: V \times C \to \mathbb{R}^+$ (e.g., hop count between meaurement points and control plane).

\para{Output.} The output comprises:
(1) $x_p \in \{0,1\}$ $\forall p \in P$: $x_p=1$ if sketch deployed at switch $p$;
(2) $y_p \in \{0,1\}$ $\forall p \in P$: $y_p=1$ if INT deployed at switch $p$;
(3) $u_p \in \{0,1\}$ $\forall p \in P$: $u_p=1$ if switch $p$ is selected ($x_p =1 \lor y_p=1$);
(4) $z_{p,c} \in \{0,1\}$ $\forall p \in P, c \in C$: $z_{p,c}=1$ if data from $p$ sent to control node $c$;
and (5) $w_f \in \{0,1\}$ $\forall f \in \mathcal{F}$: $w_f=1$ if flow $f$ is covered.

\para{Objective.} The objective refers to:
\begin{equation}
\min\ -\alpha \sum_{f \in \mathcal{F}} w_f + (1-\alpha) \sum_{\substack{p \in P \\ c \in C}} z_{p,c} \cdot \delta(p, c)
\label{eq:objective}
\end{equation}
where the user-configurable parameter $\alpha \in [0,1]$ balances the objective of maximizing coverage and that of minimizing the distances between selected switches and control plane nodes. 

\para{Constraints.} The problem has the following major constraints: Eq.2-4 bound the number of selected switches; Eq.5 presents that when a flow $f$ is covered, at least one switch on the shortest paths that connect the OD pair of $f$ should be selected; Eq.6 shows that each selected switch should send data to a control plane node; and Eq.7 limits decision variables. 

\vspace{-7pt}
{\footnotesize
\begin{align}
\text{Selected switch number: } &u_p \geq x_p \quad \forall p \in P \\
&u_p \geq y_p \quad \forall p \in P \\
&u_p \leq x_p + y_p \quad \forall p \in P \\
\text{Coverage: } &w_f \leq \sum_{p \in \mathcal{P}_f} u_p \quad \forall f \in \mathcal{F} \\
\text{Data collection: } &\sum_{c \in C} z_{p,c} = u_p \quad \forall p \in P \\
\text{0-1 decisions: } &x_p, y_p, u_p, w_f, z_{p,c} \in \{0,1\}
\end{align}}

There also exist other objectives, e.g., maximizing the number of covered switches or the number of covered links. While our focus is flow coverage, supporting other objectives is still simple, i.e., replacing Eq.1 with user-specified objectives. 

\para{Hardness}. This measurement point selection problem is NP-hard. First, according to its dual combinatorial structures, this problem reduces to the set cover problem: each programmable switch $p \in P$ corresponds to the set $S_p = \{ f \in \mathcal{F} \mid p \in \mathcal{P}_f \}$ covering the flows possibly traversing $p$. This step establishes a set system with the universe $U = \mathcal{F}$ and collection $\mathcal{S} = \{ S_p \}$, where maximizing coverage $\sum w_f$ equals minimizing uncovered flows (i.e., a standard NP-hard set cover problem). Second, the objective of minimizing distances, i.e., $\sum z_{p,c} \delta(p,c)$, forms an uncapacitated facility location subproblem, in which selected switches, i.e., $u_p=1$, are facilities, control plane nodes in $C$ are clients, and $\delta(p,c)$ are assignment costs. 

To sum up, the joint objective in Eq.1 forms a multi-objective set cover with facility location costs. In particular, $\alpha$ modulates between flow coverage and resource efficiency: when $\alpha=1$, the problem is an NP-hard set cover problem; when $\alpha=0$, the problem is an NP-hard facility location problem.  

\para{Solving}. Given the NP hardness, directly solving the problem using commercial MIP solvers such as Gurobi \cite{gurobi} and CPLEX \cite{cplex} inevitably suffers from excessively long execution time or fails to find possible solutions. In response, \sysname employs Lagrangian relaxation as follows.

We relax the coverage constraints using Lagrange multipliers \(\lambda_f \geq 0\). The Lagrangian is:

\vspace{-10pt}
{\footnotesize
\begin{align}
L(\boldsymbol{\lambda}) = &\min_{\mathbf{u},\mathbf{w},\mathbf{z}} \left[ 
-\alpha \sum_{f} w_f + (1-\alpha) \sum_{p,c} z_{p,c} \delta(p,c) \right. \nonumber \\ &\left. + \sum_{f} \lambda_f \left( \sum_{p \in \mathcal{P}_f} u_p - w_f \right)
\right] 
= \min_{\mathbf{u},\mathbf{w},\mathbf{z}} \left[ 
\underbrace{\sum_{p} u_p \left( \sum_{f: p \in \mathcal{P}_f} \lambda_f \right)}_{\text{Switch penalty}} \right. \nonumber \\ &+ \left.
\underbrace{\sum_{f} w_f (-\alpha - \lambda_f)}_{\text{Coverage reward}} +
\underbrace{(1-\alpha) \sum_{p,c} z_{p,c} \delta(p,c)}_{\text{Data collection distance cost}}
\right]
\end{align}}
\vspace{-7pt}

\noindent where $\lambda_f$ is the Lagrange multiplier for flow $f$ and quantifies the penalty weight for not covering $f$. The switch penalty refers to the penalty for selecting a switch. It represents that selecting a switch needs to pay the penalty of covering all its measurable flows. Therefore, it transforms the original coverage constraints into switch selection penalties that balance against the reward of covering flows and the distance cost of transferring data from selected switches to the control plane. 

In this context, the dual of the Lagrangian relaxation is:
\begin{align}
\max_{\lambda_f \geq 0} L(\boldsymbol{\lambda}) 
\end{align}
subject to primal constraints in Eq.2-Eq.7. 
%The solution is attained via subgradient ascent: $\lambda_f^{(t+1)} = \max\left(0, \lambda_f^{(t)} + \frac{1}{\sqrt{t}} \left( \sum_{p \in \mathcal{P}_f} u_p - w_f \right)\right)$.

\begin{algorithm}[t]
\caption{Solve \( L(\boldsymbol{\lambda}) \)}
\label{alg:solve-lagrangian}
\begin{algorithmic}[1]
\footnotesize
\Require \(\boldsymbol{\lambda}\), \(G\), \(\mathcal{F}\), \(P\), \(C\), \(\delta\), \(\alpha\) (See Table~I)
\Ensure \(L(\boldsymbol{\lambda})\), primal variables
\State Initialize \(u_p = 0\), \(w_f = 0\) \(\forall p \in P, f \in \mathcal{F}\)
\For{each switch \(p \in P\)}
  \State Compute \(\beta_p = \sum_{f: p \in \mathcal{P}_f} \lambda_f\) \Comment{{\color{aogreen}Aggregate flow penalties}}
  \State Compute \(\gamma_p = (1-\alpha) \min_{c \in C} \delta(p, c)\) \Comment{{\color{aogreen}Min distance cost}}
  \If{\(\beta_p + \gamma_p < 0\)} \Comment{{\color{aogreen}Negative reduced cost}}
    \State Set \(u_p = 1\)
    \State Assign \(z_{p,c^*} = 1\) for \(c^* = \arg\min_c \delta(p,c)\)
  \EndIf
\EndFor
\For{each flow \(f \in \mathcal{F}\)}
  \If{\(-\alpha - \lambda_f < 0\)} \Comment{{\color{aogreen}Negative coverage reward}}
    \State Set \(w_f = 1\) \Comment{{\color{aogreen}Always true for \(\lambda_f \geq 0, \alpha > 0\)}}
  \EndIf
\EndFor
\State \Return \(L(\boldsymbol{\lambda}) = \sum_p u_p (\beta_p + \gamma_p) + \sum_f w_f (-\alpha - \lambda_f)\)
\end{algorithmic}
\end{algorithm}

\begin{algorithm}[t]
\caption{Subgradient Optimization for Dual Problem}
\label{alg:subgradient}
\begin{algorithmic}[1]
\footnotesize
\Require Max iterations \(T\), step size \(\theta_t\)
\Ensure \(\max_t L(\boldsymbol{\lambda}^{(t)})\)
\State Initialize \(\lambda_f^{(0)} = 0\) \(\forall f \in \mathcal{F}\), \(t = 0\)
\While{\(t < T\)}
  \State Solve \(L(\boldsymbol{\lambda}^{(t)})\) using Algorithm \ref{alg:solve-lagrangian}
  \State Compute subgradient \(g_f^{(t)} = \sum_{p \in \mathcal{P}_f} u_p^* - w_f^*\)
  \State Update \(\lambda_f^{(t+1)} = \max\left(0, \lambda_f^{(t)} + \theta_t g_f^{(t)}\right)\)
  \State \(t \leftarrow t + 1\)
\EndWhile
\State \Return \(\max_t L(\boldsymbol{\lambda}^{(t)})\)
\end{algorithmic}
\end{algorithm}

According to the above relaxation, \sysname uses a master-slave algorithm to solve the problem: Algorithm~2 is the master that controls the optimization while Algorithm~1 is the slave that is invoked in each iteration to solve the Lagrangian subproblem. Specifically, Algorithm~2 iteratively tunes Lagrange multipliers $\boldsymbol{\lambda}$ to maximize the dual function. In each iteration $t$, it (1) inputs $\boldsymbol{\lambda}^{(t)}$ to Algorithm~1, (2) receives primal variables $u_p, w_f, z_{p,c}$ and dual bound $L(\boldsymbol{\lambda})$, (3) computes the subgradient $g_f^{(t)}$ (line~4), and updates $\boldsymbol{\lambda}^{(t+1)} = \max(0, \lambda_f^{(t)} + \theta_t g_f^{(t)})$ with $\theta_t = 1/\sqrt{t}$. 

In the step (1) of each iteration in Algorithm~2, Algorithm~1 computes optimal deployments for given $\boldsymbol{\lambda}$: for each switch $p$, it calculates the flow penalty $\beta_p = \sum_{f: p \in \mathcal{P}f} \lambda_f$ and collection cost $\gamma_p = (1-\alpha) \min_c \delta(p,c)$. If $\beta_p + \gamma_p < 0$, it selects $p$ to activate measurement, i.e., $u_p=1$, and assigns $p$ to the nearest control plane node, i.e., $z_{p,c^*}=1$, and then marks all flows covered, i.e., $w_f=1$, since $-\alpha - \lambda_f < 0$ always holds.

\begin{theorem}[Weak Duality]
The optimal dual value \(d^* = \max_{\boldsymbol{\lambda}} L(\boldsymbol{\lambda})\) is a lower bound on the primal optimal value \(p^*\):
\[
d^* \leq p^*
\]
\end{theorem}

\begin{proof}
For any feasible primal solution \((\mathbf{u},\mathbf{w},\mathbf{z})\) and \(\boldsymbol{\lambda} \geq 0\):
\[
\begin{aligned}
L(\boldsymbol{\lambda}) &\leq -\alpha \sum_f w_f + (1-\alpha) \sum_{p,c} z_{p,c} \delta(p,c) \\
&+ \sum_f \lambda_f \left( \sum_{p \in \mathcal{P}_f} u_p - w_f \right) \\
&\leq -\alpha \sum_f w_f + (1-\alpha) \sum_{p,c} z_{p,c} \delta(p,c) \\
&= p_{\text{obj}} \quad \text{(since } \sum_{p \in \mathcal{P}_f} u_p - w_f \geq 0 \text{ by Eq.5)}
\end{aligned}
\]
where $p_{\text{obj}}$ is the primal objective value for a feasible solution in the measurement point selection problem. Thus, \(d^* \leq p^*\).
\end{proof}

\begin{theorem}[Optimality Gap]
Let \((\mathbf{u}^*,\mathbf{w}^*,\mathbf{z}^*)\) be the primal optimal solution. The optimality gap satisfies:
\[
p^* - d^* \leq \sum_{f \in \mathcal{F}} \lambda_f^* \left( \sum_{p \in \mathcal{P}_f} u_p^* - w_f^* \right)
\]
where \(\boldsymbol{\lambda}^*\) are the optimal dual multipliers.
\end{theorem}

\begin{proof}
From complementary slackness:
\[
\begin{aligned}
p^* - d^* &= \left[ -\alpha \sum_f w_f^* + (1-\alpha) \sum_{p,c} z_{p,c}^* \delta(p,c) \right] - L(\boldsymbol{\lambda}^*) \\
&= \sum_f \lambda_f^* \left( w_f^* - \sum_{p \in \mathcal{P}_f} u_p^* \right) \\
&\leq 0 \quad \text{(because } w_f^* \leq \sum_{p \in \mathcal{P}_f} u_p^* \text{ and } \lambda_f^* \geq 0\text{)}
\end{aligned}
\]
The gap equals the sum of violated coverage constraints scaled by \(\lambda_f^*\).
\end{proof}

\begin{theorem}[Time Complexity]
Given a network $G = (V, E)$, programmable switches in $P \subseteq V$, control plane nodes in $C \subseteq V$, flows in $\mathcal{F}$, and flow-switch incidence sets $\{\mathcal{P}_f\}$, the time complexity of Algorithms 1-2 is $O(T \cdot |P| \cdot (d_{\max} + |C|))$, where $T$ is the iteration count, $d_{\max} = \max_{p \in P} |\{f : p \in \mathcal{P}_f\}|$, and $|C|$ is the number of control plane nodes.
\end{theorem}

\begin{proof}
The complexity is contributed by all the $T$ iterations of Algorithm 2, each of which invokes Algorithm 1. For Algorithm 1, it invokes: (1) for each $p \in P$, it computes $\beta_p = \sum_{f: p \in \mathcal{P}_f} \lambda_f$ in $O(d_p)$ time, where $d_p = |\{f : p \in \mathcal{P}_f\}| \leq d_{\max}$, (2) for each $p \in P$, it computes $\gamma_p = (1-\alpha) \min_{c \in C} \delta(p,c)$ in $O(|C|)$ time, (3) for each $p \in P$, it sets decision variables in $O(1)$ time. Thus, the time complexity of Algorithm 1 is $O(|P| \cdot (d_{\max} + |C|))$. For Algorithm 2, each iteration invokes: (1) $O(|P| \cdot (d_{\max} + |C|))$ Algorithm 1, (2) computing subgradient $g_f = \sum_{p \in \mathcal{P}_f} u_p$ for all $f \in \mathcal{F}$ in $O\left(\sum_f |\mathcal{P}_f|\right) = O(|P| \cdot d_{\max})$ time, and (3) updating multipliers in $O(|\mathcal{F}|) \subseteq O(|P| \cdot d_{\max})$ time. Thus, Algorithm~2 has $T$ iterations, yielding $O(T \cdot |P| \cdot (d_{\max} + |C|))$.
\end{proof}

% \begin{theorem}[Algorithm~1 Complexity]
% The time complexity of Algorithm~1 is $\mathcal{O}(|P| \cdot (d_{\max} + |C|))$, where $|P|$ is the number of switches, $d_{\max} = \max_{p \in P} |\{f : p \in \mathcal{P}_f\}|$, and $|C|$ is the number of control plane nodes.
% \end{theorem}

% \begin{proof}
% For each switch $p$, Algorithm 1 performs (1) $\beta_p$ computation (line~3), which complexity is $\mathcal{O}(d_p)$ ($d_p$ counts how many flows can potentially be measured by $p$), (2) $\gamma_p$ computation (line~4), which complexity is $\mathcal{O}(|C|)$, and (3) making decisions (lines~5-7), which complexity is $\mathcal{O}(1)$. To sum up, the 

% Summing over $p$: $\sum_p \mathcal{O}(d_p) = \mathcal{O}(|P| \cdot d_{\max})$. Adding coverage marking $\mathcal{O}(|\mathcal{F}|)$ and noting $|\mathcal{F}| \leq |P| \cdot d_{\max}$, total complexity is $\mathcal{O}(|P| \cdot (d_{\max} + |C|))$.
% \end{proof}

% \begin{theorem}[Algorithm 2 Complexity]
% The time complexity for $T$ iterations of Algorithm 2 is $\mathcal{O}(T \cdot |P| \cdot (d_{\max} + |C|))$.
% \end{theorem}

% \begin{proof}
% Each iteration involves:
% \begin{itemize}
%   \item Algorithm 1 call: $\mathcal{O}(|P| \cdot (d_{\max} + |C|))$
%   \item Subgradient computation: $\mathcal{O}(|\mathcal{F}| \cdot s_{\max}) \subseteq \mathcal{O}(|P| \cdot d_{\max})$ (since $\sum_f |\mathcal{P}_f| = \sum_p d_p$)
%   \item Multiplier update: $\mathcal{O}(|\mathcal{F}|)$
% \end{itemize}
% Total per iteration: $\mathcal{O}(|P| \cdot (d_{\max} + |C|))$. For $T$ iterations: $\mathcal{O}(T \cdot |P| \cdot (d_{\max} + |C|))$.
% \end{proof}

% to the coverage constraints $w_f \leq \sum_{p \in \mathcal{P}_f} u_p$ by introducing multipliers $\lambda_f \geq 0$. The Lagrangian dual function is:

% {\footnotesize
% \begin{equation}
% L(\boldsymbol{\lambda}) = \min \left[ 
% \sum_{p} u_p \left( \sum_{f: p \in \mathcal{P}_f} \lambda_f \right) + \sum_{f} w_f (-\alpha - \lambda_f) + (1-\alpha) \sum_{p,c} z_{p,c} \delta(p,c)
% \right],
% \end{equation}}

% \noindent subject to switch usage (Eq. \ref{eq:u_x}--\ref{eq:u_sum}), control assignment (Eq. \ref{eq:control}), and binarity constraints. This decouples into: (1) Switch activation: for each $p$, set $u_p = 1$ if $\sum_{f: p \in \mathcal{P}_f} \lambda_f + (1-\alpha) \min_c \delta(p,c) < 0$, and assign $z_{p,c^*}=1$ for nearest $c^*$; (2) Coverage: set $w_f=1$ $\forall f$ (since $-\alpha - \lambda_f < 0$). We maximize $L(\boldsymbol{\lambda})$ via subgradient optimization: initialize $\lambda_f=0$; update $\lambda_f^{(t+1)} = \max\left(0, \lambda_f^{(t)} + \frac{1}{\sqrt{t}} \left( \sum_{p \in \mathcal{P}_f} u_p^* - w_f^* \right)\right)$ for $T$ iterations.

% \textbf{Optimality Analysis.}
% By weak duality, $d^* = \max_{\boldsymbol{\lambda}} L(\boldsymbol{\lambda}) \leq p^*$ for all $\boldsymbol{\lambda} \geq 0$, as $L(\boldsymbol{\lambda}) \leq$ primal objective when constraints hold. The gap $p^* - d^*$ is bounded by $\sum_f \lambda_f^* \left( \sum_{p \in \mathcal{P}_f} u_p^* - w_f^* \right)$, vanishing when coverage constraints are tight. Primal recovery achieves $\epsilon$-optimal solutions with $p_{\text{found}} \leq p^* + \epsilon$ via switch activation heuristics.


\section{Loss-Free Measurement Data Collection}\label{collection}

\para{Overview}. After making decisions on which switches to measure traffic, \sysname determines which network paths to safely transfer measurement data from switches to control plane nodes without incurring congestion. It executes a two-step workflow. First, it estimates the worst-case sending rate of sketch data and INT data in each switch. Second, according to its estimates, it picks paths to transfer data without congestion.

\para{Estimating worst-case sending rates}. \sysname models both sketch and INT data. For sketch data, the worst-case sending rate $\gamma_k^{\text{sketch}}$ for sketch $k$ is determined by:

\vspace{-4pt}
\begin{align}
\gamma_k^{\text{sketch}} = \frac{S_k}{T_k}
\end{align}

\noindent where $S_k$ and $T_k$ refer to the sketch size in bytes and the collection window in seconds, respectively. This equation denotes that the data recorded in sketch counter arrays are periodically flushed to the control plane at the end of each window, e.g., for the sketch with 10\,MB memory, a measurement window of 1\,ms corresponds to a $\gamma_k^{\text{sketch}}=\frac{10^{6}\,\text{bytes}}{10^{-3}\,\text{s}}=8\,\text{Gbps}$.  

For INT data, the worst-case rate $\gamma_p^{\text{INT}}$ at switch $p$ is:

\vspace{-4pt}
\begin{align}
\gamma_p^{\text{INT}} = \left( \frac{C_p \times \phi}{\mu} \right)\times B_{\text{INT}} 
\end{align}

\noindent where $C_p$ refers to the bandwidth capacity of each link while $\phi$ is the maximum possible fraction of link bandwidth that can be consumed by small flows. $\phi$ can be obtained by analyzing historical traffic data \cite{roy2015inside}. $C_p\times \phi$ denotes the maximum possible byte rate contributed by small flows on a link. Moreover, since INT overheads are per-packet, \sysname converts the byte rate of $C_p\times \phi$ to the packet rate to estimate INT overheads, where $\mu$ denotes the average packet size in small flows. As such, the maximum packet rate of small flows becomes $\frac{C_p \times \phi}{\mu}$. So Eq.10 estimates the bandwidth overheads consumed by INT headers per second, where $B_{\text{INT}}$ is the size of INT headers per packet. 

To sum up, the maximum possible rate of sending measurement data at switch $p$ is estimated as

\vspace{-4pt}
\begin{align}
\Gamma_p^{\text{total}} = \sum_{k \in \mathcal{K}_p} \gamma_k^{\text{sketch}} + \gamma_p^{\text{INT}}
\end{align}

\noindent where $\mathcal{K}_p$ is the set of sketches at $p$. 

%$\Gamma_p^{\text{total}}$ represents the peak load at $p$.

\para{Reinforcement learning (RL) for dynamic path selection}. Given the worst-case estimates in $\{\Gamma_p^{\text{total}}\}$, \sysname formulates path selection with a constrained Markov decision process (CMDP) to avoid saturating network paths, i.e.

\vspace{-4pt}
\begin{align}
\min_{\pi}  & \quad \mathbb{E}\left[\sum_{t=0}^T \sum_{e \in E} q_e^t \right] \\
\text{s.t.} & \quad \sum_{c \in C} \pi_{p,c} = 1 \quad \forall p \in P  \\
            & \quad \Gamma_e^{\text{meas}} \leq \tau_e C_e - \Gamma_e^{\text{data}} \quad \forall e \in E 
\end{align}

\noindent where $q_e^t$ represents the queue depth in bytes on link $e$ at time $t$; $\pi_{p,c}$ is the fraction of measurement data streams from switch $p$ to control plane node $c$; $\Gamma_e^{\text{meas}}$ quantifies the measurement data streams on link $e$, i.e. 

\vspace{-4pt}
\begin{align}
    \Gamma_e^{\text{meas}} = \sum_{p} \sum_{c} \pi_{p,c} \cdot \Gamma_p^{\text{total}} \cdot \mathbb{I}_{e \in \mathcal{P}_{p,c}}
\end{align}

\noindent where $\mathbb{I}_{e \in \mathcal{P}_{p,c}}=1$ if link $e$ belongs to path $\mathcal{P}_{p,c}$; $\Gamma_e^{\text{data}}$ quantifies data plane traffic on $e$; and $\tau_e$ represents the safety threshold that avoids congestion (e.g., 0.8 of the link capacity).

Here, the inputs include $q_e^t$, $C_e$, $\Gamma_p^{\text{total}}$, $\mathbb{I}_{e \in \mathcal{P}{p,c}}$, $\Gamma_e^{\text{data}}$, and $\tau_e$. \sysname obtains them as follows: $q_e^t$ is collected by INT; $C_e$ is the link capacity that is already known in topology information; $\Gamma_p^{\text{total}}$ is computed by Eq.12; $\mathbb{I}_{e \in \mathcal{P}{p,c}}$ is calculated via shortest path algorithms such as Dijkstra; $\Gamma_e^{\text{data}}$ can be direcly obtained using existing techniques such as sFlow \cite{sFlow} and NetFlow \cite{netflow}; and $\tau_e$ is user-configurable. 

Next, the output of \sysname includes (1) which paths, i.e., $\mathcal{P}_{p,c}$, to transfer measurement data from switch $p$ to control plane node $c$, and (2) split ratios, i.e., $\pi_{p,c}$, that decide how to allocate measurement traffic across these paths. 

Then \sysname solves the CMDP as follows. First, it trains the policy $\pi_\theta$ via offline constrained policy optimization (CPO) \cite{achiam2017constrained}. Such training employs a simulated network environment with the following components: 
\begin{itemize}[leftmargin=*,noitemsep]
\item \textit{State}: $s_t = \left( \{ \Gamma_e^{\text{data}}/C_e \}, \{q_e^t\}, \{\Gamma_p^{\text{total}}\} \right)$.
\item \textit{Action}: (1) paths in $\{\mathcal{P}_{p,c}\}$, and (2) splitting ratios $\pi_{p,c}$.
\item \textit{Reward}: $r_t = -\sum_{e \in E} q_e^t - \eta \cdot \mathbb{I}_{\text{violation}}$, where $-\sum_{e \in E} q_e^t$ encourages RL to minimize network congestion, and $\eta \cdot \mathbb{I}_{\text{violation}}$ penalizes constraint violations: $\eta$ is a large penalty weight (e.g., $10^6$) and $\mathbb{I}_{\text{violation}}=1$ if $\Gamma_e^{\text{data}} + \Gamma_e^{\text{meas}} > \tau C_e \text{ for any } e$. 
\end{itemize}

\noindent With these components, \sysname maximizes $r_t$ while satisfying Eq.15 via Lagrangian relaxation. After $O(10^5)$ training steps with real traffic traces (e.g., CAIDA), $\pi_\theta$ is converged and is deployed for online inference.

\begin{algorithm}[t]
\caption{Congestion-free path selection}
\begin{algorithmic}[1]
\Require Trained policy network $\pi_\theta$, network $G$, 
         control plane nodes in $C$, programmable switches in $P$, 
         worst-case rates $\{\Gamma_p^{\text{total}}\}_{p \in P}$, 
         safety threshold $\tau$
\Ensure Selected paths in $\{\mathcal{P}_{p,c}\}$ and splitting ratios in $\{\pi_{p,c}\}$

\State Initialize $\{\mathcal{P}_{p,c}\} \gets \textsc{k-ShortestPaths}(G, P, C)$ %\Comment{Multiple path options}
\State Initialize $\pi_{p,c} \gets \frac{1}{|C|}$ \Comment{{\color{aogreen}Uniform initial splitting}}

\Loop \Comment{{\color{aogreen}At each time step $t$}}
  \State Collect $\Gamma_e^{\text{data}}$ and $q_e^t$
  \State Build state $s_t \gets \left( \{ \frac{\Gamma_e^{\text{data}}}{C_e} \}, \{q_e^t\}, \{\Gamma_p^{\text{total}}\} \right)$
  \State Get action $a_t \gets \pi_\theta(s_t)$ and yield $(\{\mathcal{P}_{p,c}\}, \{\pi_{p,c}\})$ 
  \State Compute $\Gamma_e^{\text{meas}} \gets \sum_{p} \sum_{c} \pi_{p,c} \cdot \Gamma_p^{\text{total}} \cdot \mathbb{I}_{e \in \mathcal{P}_{p,c}}$
  \State $\pi_{p,c} \gets \pi_{p,c} \cdot \min\left(1, \frac{\tau C_e - \Gamma_e^{\text{data}}}{\Gamma_e^{\text{meas}} + \epsilon}\right)$ for $e \in \mathcal{P}_{p,c}$
  \State $\pi_{p,c} \gets \frac{\pi_{p,c}}{\sum_c \pi_{p,c}} \quad \forall p \in P$ \Comment{{\color{aogreen}Normalization}}
  \State Deploy $\mathcal{P}_{p,c}$ and $\pi_{p,c}$ on switches
\EndLoop
\end{algorithmic}
\end{algorithm}

With $\pi_\theta$, \sysname creates a RL framework in Algorithm~3. Initially (lines~1-2), it precomputes $k$ candidate paths in $\{\mathcal{P}_{p,c}\}$ between all $p$-$c$ pairs and initializes uniform splitting ratios, i.e., $\pi_{p,c} = 1/|C|$. At each timestep $t$ (loop in line~3), $q_e^t$ and $\Gamma_e^{\text{data}}$ are collected and transformed into state $s_t$ (lines~4-5). Then the RL policy $\pi_\theta$ yields path selections and raw splitting ratios (line~6). 
After computing $\Gamma_e^{\text{meas}}$ via Eq.16 (line~7), \sysname enforces bandwidth constraints via scaling and normalization (lines~8-9). 
Finally, selected paths $\mathcal{P}_{p,c}$ and splitting ratios $\pi_{p,c}$ are deployed onto data plane switches, e.g., via rule installation (line~10). In short, this algorithm dynamically optimizes path utilization to prevent congestion during measurement data collection.

%Andante employs a reinforcement learning (RL) framework to dynamically select (1) \textit{which paths} and (2) \textit{split ratios} for transferring measurement data from programmable switches ($p \in P \subseteq V$) to control plane nodes ($c \in C \subseteq V$) without congestion. The algorithm takes four key inputs (Line 1): (1) a \textit{trained policy network} $\pi_\theta$ (obtained via offline CPO training) that jointly outputs path selections $\mathcal{P}_{p,c}$ and splitting ratios $\pi_{p,c}$, (2) network topology $G = (V, E)$, (3) fixed worst-case rates $\{\Gamma_p^{\text{total}}\}_{p \in P}$, and (4) safety threshold $\tau$. 

\section{Example of \sysname}

We present a simple example to explain how \sysname works. 

\para{Settings}. We consider a network comprising a triangle of three programmable switches, $p_1$, $p_2$, $p_3$. All links are 10-Gbps. The control plane is directly connected to $p_2$. Five flows traverse the network: a video flow $f_1$ with the OD pair of ($p_1$,$p_2$), a RPC flow $f_2$ with the OD pair of ($p_1$,$p_3$), a database flow $f_3$ with the OD pair of ($p_2$,$p_3$), an IoT flow $f_4$ with the OD pair of ($p_3$,$p_1$), and a backup flow $f_5$ with the OD pair of ($p_3$,$p_2$). $f_1$, $f_3$, and $f_5$ are large flows while small flows are $f_2$ and $f_4$. The safety threshold $\tau$ is 0.8 of link capacity. 

\para{Measurement point selection}. \sysname selects $p_2$ to deploy sketches. For large flows, $p_2$ covers all large flows, minimizing resource usage and data collection distance. For small flows, $p_1$ monitors $f_2$ while S3 monitors $f_4$. This decision achieves 100\% coverage while reducing the number of activated switches by 33\% compared to full INT deployment. 

%Thus, it balancing the tradeoff between coverage efficiency and proximity constraints through penalty weight optimization.

\para{Measurement data collection}. Consider the state $s_t$: (1) link utilizations: $p_1\leftrightarrow p_3=20$\%, $p_2\leftrightarrow p_3=70$\%, $p_1\leftrightarrow p_2=20$\%; (2) queue depths: 40\,KB in each switch; (3) worst-case measurement data rates: $\Gamma_{p_1}^{\text{total}}=3$\,Gbps, $\Gamma_{p_2}^{\text{total}}=1$\,Gbps, and $\Gamma_{p_3}^{\text{total}}=2$\,Gbps. The RL policy $\pi_{\theta}$ in \sysname decides to (1) deliver 100\% $p_1$'s INT data using the path $p_1\leftrightarrow p_2$, and (2) deliver 100\% $p_3$'s INT data using the path $p_3\rightarrow p_1\rightarrow p_2$. Otherwise, if $p_3$ sends INT data to $p_2$, the link $p_2\leftrightarrow p_3$ suffers from the load of 8\,Gbps, which violates the safety threshold $\tau$: $\Gamma_{p_3}^{\text{total}}+70$\%$\times 10$\,Gbps$=9$\,Gbps $> \tau\times 10$\,Gbps$=8$\,Gbps. This also avoids congesting other paths, e.g., for the path $p_1 \leftrightarrow p_2$, $\Gamma_{p_1}^{\text{total}}+\Gamma_{p_3}^{\text{total}}+20$\%$\times 10$\,Gbps$=7$\,Gbps $< \tau\times 10$\,Gbps$=8$\,Gbps, preserving the safety threshold and avoid data loss. 

% Queue depths: S1→S3=120 KB (congested), others=40 KB

% Worst-case rates: Γ<sub>S1</sub><sup>total</sup>=0.3 Gbps (INT for F2), Γ<sub>S3</sub><sup>total</sup>=0.15 Gbps (INT for F4)
% RL policy $\pi_θ$ outputs:

% Divert 70\% of S1's INT data through S1→S2→S3 (avoiding congested S1→S3)

% Route 100\% of S3's INT data via S3→S2 (direct path underutilized)
% Congestion safeguard scales ratios to ensure:

% S1→S2: 60\% + (0.3×0.7×10\%)=62.1\% < 80\%

% S3→S2: 82% + 1.5%=83.5% → over threshold! → dynamically reroutes 30% to S3→S1→S2
% Final deployment achieves zero packet loss with 43 μs INT latency.

%Consider that the current link utilizations: S1$\rightarrow$S2=65\%, S2$\rightarrow$S3=72\%, S3$\rightarrow$S2=60\%; queue depths are 50\,KB; and the worst-case measurement data rates: 0.5\,Gbps from S1, 7\,Gbps from S2, and 0.3\,Gbps from S3. The decisions of \sysname are: 100\% of S1's INT data (measuring $f_2$) and S3's INT data (measuring $f_4$) are transferred through the path S1$\rightarrow$S2 and the path S3$\rightarrow$S2, respectively. Therefore, none of links will exceed the safety threshold.  

%The RL policy processes real-time telemetry (link utilizations: S1→S2=65\%, S2→S3=72\%, S3→S2=60\%; queue depths: 50\,KB uniform) and worst-case measurement rates (S1:0.5Gbps, S2:7Gbps, S3:0.3Gbps). It outputs direct-path routing: 100\% of S1's INT data (F2) and S3's INT data (F4) through S1→S2 and S3→S2 respectively. The congestion safeguard validates projected utilization (S1→S2: 65\% + 5\% = 70\% $<$ 80\% threshold; S3→S2: 60\% + 3\% = 63\% $<$ 80\%), confirming loss-free transmission without scaling. P4 flow rules enforce these paths dynamically during traffic spikes.





\section{Evaluation}\label{eval}

In this section, we perform testbed experiments to evaluate \sysname. We highlight our findings as follows.

\begin{itemize}[leftmargin=*]
%
    \item Compared to a wide spectrum of baselines, \sysname provides $3.3\times$ higher accuracy to various applications. Compared to sketch-only baselines, \sysname achieves at least $3.3\times$ higher accuracy when measuring small flows. Compared to INT-only baselines, \sysname reduces at least $3.3\times$ overheads in network bandwidth and control plane resources. Compared to hybrid baselines, \sysname improves accuracy by 3.3$\times$. 
%
    \item \sysname offers the near-optimal measurement point selection that improves the flow coverage by $3.7\times \sim 3.8\times$ and reduces the distance between switches and control plane nodes by $3.7\times \sim 3.8\times$ when compared to strawman approaches such as heuristics. It also maintains timeliness that quickly yields decisions within a few milliseconds. 
%
    \item \sysname eliminates network congestion via its loss-free measurement data collection. Its dynamic path selection prevents high link utilization in all cases, while existing solutions fail to avoid data loss when dynamics occur. 
%
\end{itemize}

\subsection{Experimental Setup}\label{setup}

\para{Implementation}. Our implementation comprises optimization algorithms and a controller. For the optimization algorithms, we implement the Lagrangian relaxation algorithms for measurement point selection (\S\ref{selection}) in C++ using the Gurobi API \cite{gurobi}. The RL algorithm for congestion-free data collection (\S\ref{collection}) is implemented using PyTorch with the CPO library \cite{pytorch}. Next, we implement a controller in Python that translates optimization outputs (i.e., \(x_p\), \(y_p\), \(z_{p,c}\) variables) into switch configurations. Specifically, it configures: (1) sketch deployment and INT activation on selected switches via data plane program deployment \cite{chen2020speed,gao2020lyra}, and (2) data routing rules inside match-action tables to direct measurement data to target paths. The controller uses the Tofino SDE \cite{tofino2} to load configurations on switches.

\para{Testbed and simulator}. We construct a real testbed comprising six \(64 \times 400\)\,Gbps Tofino2 programmable switches \cite{tofino2}. These switches are interconnected within a two-level fat-tree topology via 100-Gbps links. Also, at the network edge, we establish a cluster of high-performance servers, each offering 36-core Intel Xeon Gold 6240C 2.60\,GHz CPU and 128-GB RAM, as the control plane. For traffic generation, we connect eight servers that runs PktGen-DPDK \cite{pktgen} at line rate to the network. Next, due to limited testbed scale, we implement a C++ simulator to simulate large-scale topologies that models: (1) switch pipelines with resource constraints \cite{jose2015compiling}, (2) link bandwidths and latency, and (3) switch and control-plane performance statistics based on real-world measurements. The simulator accepts real traffic traces such as CAIDA \cite{caida} and topology files as input.

\para{Topologies and workloads}. We simulate two typical classes of topologies \cite{anup2022hetero,chen2024eagle}. (1) Wide-area networks (WANs). We select five WANs, AboveNet (20 switches), Internet2 (42 switches), ForthNet (62 switches), Bestel (83 switches), and Interroute (109 switches), from the Internet topology zoo \cite{knight2011internet}. We denote these topologies with T1-T5. We set the transmission latency of each link to be randomly distributed between 10\,ms to 50\,ms while setting all switches to be programmable. (2) Data center networks (DCNs). For DCNs, we consider the fat-tree networks and vary the pod number: C1-C5 that use 16, 20, 24, 32, and 48 pods, respectively, e.g., C5 has 2880 switches and 55296 links. We set the link latency to be randomly varied between 1\,$\mu$s to 10\,$\mu$s. We randomly pick 30\% of switches to be programmable based on recent statistics \cite{telecom}. For each programmable switch, we set its configurations based on Tofino switches \cite{jose2015compiling}.

For control plane nodes, we leverage two distinct strategies for WANs and DCNs, respectively. For WANs, we employ the technique that optimizes the placement of control plane nodes in WANs \cite{heller2012controller}. For DCNs, we allocate a distinct control plane node to each pod of fat-tree networks. 

We choose corresponding workloads that match WANs and DCNs. For WANs, we choose a CAIDA trace \cite{caida} that lasts for one minute and comprises around 26\,M packets. For DCNs, we use the complete IMC DCN traffic trace \cite{benson2010network}. Each flow's OD pair is randomly selected among edge switches. 


%We realize them with CM, CS, ES, FR, HP, and UM, respectively. We realize EE with FR, MARC, and UM, respectively.

\para{Baselines}. We compare \sysname with five types of techniques. 

\begin{itemize}[leftmargin=*]
%
    \item[1] \textbf{Sketches only}. We implement six state-of-the-art sketches, including the count-min sketch (CM) \cite{cormode2005improved}, the count sketch (CS) \cite{charikar2004finding}, the elastic sketch (ES) \cite{yang2018elastic}, FlowRadar (FR) \cite{li2016flowradar}, HashPipe (HP) \cite{sivaraman2017heavy}, and UnivMon (UM) \cite{liu2016one}. We also refer to the theoretical bounds in their original papers to configure their parameters to maximize accuracy. % sketches = c++
%
    \item[2] \textbf{INT only}. We consider three classic INT techniques, including the original INT \cite{int}, PINT \cite{ben2020pint}, and DeltaINT \cite{sheng2021deltaint}. We implement them based on their open-source implementations. % int, pint (sampling rate = 0.X), deltaint (see paper)
%
    \item[3] \textbf{Hybrid sketch+INT}. Recall from \S\ref{hybrid}, we implement the open-source SketchINT \cite{yang2023sketchint} and LightGuardian \cite{zhao2021lightguardian}. 
%
    \item[4] \textbf{Measurement point selection algorithms}. We choose three algorithms for selecting measurement points, i.e., MTP \cite{chen2021mtp} that aims to minimize the number of occupied measurement points to place sketches and INT, SPEED \cite{chen2020speed} that optimizes both resources and performance, and a Greedy-based heuristic that prioritizes the switches with more resources. 
%
    \item[5] \textbf{Data collection path selection algorithms}. We use three algorithms for selecting the paths of transferring measurement data, i.e., Escala \cite{liu2022escala} that selects the paths with lower latency, a randomized algorithm that randomly chooses its paths, and INT-path \cite{pan2019int} that adopts a depth-first search (DFS) heuristic to cover as many measurement points as possible.
%
\end{itemize}

For a fair comparison, we set the top three types of techniques to select the same measurement points and paths chosen by the algorithms of \sysname. We fix the amount of switch memory allocated to each sketch or INT to 10\,MB, which is close to the maximum capacity of a switch \cite{gupta2018sonata}. 
Moreover, to quantify the solution quality of the last two types of techniques, we also use Gurobi \cite{gurobi} to find the optimal decisions (i.e., OPT). 

\para{Applications}. We consider three types of network management applications based on their differences on queries.

%(1) volumetric applications, i.e., heavy hitter detection (HH) \cite{huang2017sketchvisor}, superspreader detection (SS) \cite{tang2019mv}, DDoS flow detection (DF) \cite{liu2021jaqen}, and per-flow packet counting (PC) (2) aggregated applications, i.e., entropy estimation (EE) \cite{liu2016one}, and (3) troubleshooting applications, i.e., path monitoring (PM) \cite{ben2020pint,sheng2021deltaint}. According to the literature, we set the thresholds of HH to 10K packets and set the threshold of SS to 0.5\% of the total number of IP addresses. Also, for EE, we compute the entropy of measured flow distributions based on \cite{liu2016one}. These applications run on the control plane, which installs an instance on each server to collectively handle input data. 

% \noindent With the above terms, we classify network management applications into three classes based on their differences on queries.

\begin{itemize}[leftmargin=*]
%
    \item Volumetric applications include heavy hitter detection (HH) \cite{huang2017sketchvisor}, superspreader detection (SS) \cite{tang2019mv}, DDoS flow detection (DF) \cite{liu2021jaqen}, and per-flow packet counting (PC) \cite{huang2017sketchvisor} and query the size of each flow. 
%
    \item Aggregated applications include entropy estimation (EE) \cite{liu2016one} and flow size distribution (FD) \cite{chen2021mtp} that aggregate statistics to summarize all flows. %For FD, we set the flow identifier to two-tuple and five-tuple, respectively.
%
    \item Troubleshooting applications, e.g., latency monitoring (LM) and congestion control (CC) \cite{ben2020pint,sheng2021deltaint}, query per-flow metadata (i.e., timestamps and queue lengths).
%
\end{itemize}

\noindent According to their differences, we use different baselines: (1) Volumetric applications focus on large flows while aggregated applications consider all flows. As such, we compare \sysname with sketches-only, INT-only, and hybrid baselines. (2) Troubleshooting applications rely on INT data, such that we only compare \sysname with INT-only and hybrid baselines. For the solutions that combine both sketches and INT, we use the sketch that yields the best results for a fair comparison. 

\para{Metrics}. We use the precision, \emph{tp}/(\emph{tp}+\emph{fp}), the recall, \emph{tp}/(\emph{tp}+\emph{fn}), and the F1 score, 2\emph{tp}/(2\emph{tp}+\emph{fp}+\emph{fn}), where \emph{tp}, \emph{fp}, and \emph{fn} denote the number of true positives, false positives, and false negatives. 

\subsection{Application-Level Benefits of \sysname}

\para{\sysname brings accuracy and efficiency benefits to diverse applications}. 
In our testbed, we use \sysname and baselines to measure traffic, respectively. We continuously replay CAIDA traces \cite{caida} to our testbed and set the OD pair of each flow to random edge switches. The control plane runs the applications in \S\ref{setup} to identify the flows of their interest based on collected data. We compute the accuracy by comparing measured data with the ground truth obtained by analyzing input traces. 

% figure a-e: x = f1 + recall + precision, y = ratio (%), lines = monplan, sketch (cm,cs,...), int, pint, deltaint
% trace: caida 2018
% figure a = HH, figure b = SS, figure c = DF, figure d = EE, figure e = PM
% each figure occupies a complete line

\noindent \emph{(1) Evaluation with volumetric applications}

According to the literature, we set the thresholds of HH to 10K packets, set the threshold of SS to 0.5\% of the total number of IP addresses, and set the threshold of DF to 0.5\% of the total number of five-tuples. PC considers per-flow packet counts. 

For HH, SS, DF, in Figure~4, \sysname achieves $3.3\times$ higher accuracy. This advantage originates from co-designing sketches and INT: sketches enable accurate measurement for large flows while INT mitigate the errors of measuring small flows, while using only sketches or INT alone drops accuracy: small flows in sketches suffer from hash collisions with large flows, while direct INT brings congestion and data loss. 

%Next, we fix the threshold of distinguishing large and small flows to 10K packets and quantify the errors of measuring small flows. 
For PC, in Figure~5, \sysname achieves near-optimal accuracy because it employs fully accurate INT for small flows, which outperforms existing sketches by $3.3\times$.  

% figure a: x = sketches, y = f1, lines = monplan, sketches-only
% figure b: x = sketches, y = recall, lines = monplan, sketches-only
% trace: small flows in caida 2019=8
% each figure occupies a half line (two figures per line)

Figure~6 presents the overheads of \sysname when transfering data to the control plane. Compared with INT, \sysname reduces the rate of transferring data by two orders of magnitude when we manually increase traffic rate from 200\,Gbps to 1\,Tbps. Such a benefit comes from our design that only small flows trigger INT, thus limiting overheads and avoids data loss. 

% figure a (see excalibur paper): x = 100Gbps - 1Tbps, y = INT rate, lines = monplan, int, pint, deltaint 
% figure b: x = 100Gbps - 1Tbps, y = loss rate, lines = monplan, int, pint, deltaint 
% note: link 100 Gbps, estimate measurement data sending rate and compute loss rate
% each figure occupies a half line (two figures per line)

\noindent \emph{(2) Evaluation with aggregated applications}

For EE, we only consider FR and UM since only those two sketch-only baselines support EE. Figure~7 shows that \sysname reduces its estimation errors and produces an entropy near the true one. Again, this improvement comes from the accuracy of measuring small flows via INT in \sysname. 

For FD, we change the flow identifier to two-tuple and five-tuple, respectively. In Figure~8, \sysname produces near-optimal distributions because it well preserves the high accuracy of per-flow packet counts. In contrast, other baselines suffer from non-trivial accuracy loss due to the miss of small flows or data loss incurred by congestions during measurement data transmission. 

\noindent \emph{(3) Evaluation with troubleshooting applications}

For LM, 

For CC, 

\subsection{Microbenchmarks}

Next, to quantify the effectiveness of \sysname, we stress-test \sysname's algorithms with large-scale simulations. 

\para{\sysname achieves near-optimal measurement point selection}. We invoke \sysname and the comparison solutions to select measurement points in WANs and DCNs, respectively. Figure~6 presents the flow coverage of decisions made by each technique. It indicates that \sysname covers most flows even with unknown routing information, while other techniques fail to provide high coverage. Our algorithms also optimize the distances between switches and control plane nodes, yielding near-optimal results. 

We also measure the execution time of \sysname. Figure~6 indicates that through Lagrangian relaxation, \sysname reduces the time by orders of magnitude when compared with OPT. 

% set the number of flows to the trace used in WANs or DCNs
% set flows with random od pairs
% figure a: x = topologies, y = coverage, lines = monplan, mtp, speed, greedy, OPT (Gurobi)
% figure b: x = topologies, y = avg. distance (# hops), lines = monplan, mtp, speed, greedy, OPT (Gurobi)
% figure c: x = topologies, y = time, lines = monplan, mtp, speed, greedy, OPT (Gurobi)

\para{\sysname achieves near-optimal data collection path selection}. We invoke \sysname and the comparison solutions to select network paths for transferring measurement data, respectively. Figure~7 computes the number of congested links incurred by the decisions made by each technique. \sysname prevents link congestions in its collection, while other techniques bring non-trivial congestions, leading to high data loss rates of 23$\sim$33\%. Also, its RL algorithm maintains timeliness, which selects paths within a few milliseconds. Hence, it is acceptable for handling runtime dynamics with loss-free measurement data collection. 

%We also measure the execution time of \sysname. Figure~6 indicates that through Lagrangian relaxation, \sysname reduces the time by orders of magnitude when compared with OPT. 

% set the number of flows to the trace used in WANs or DCNs
% set flows with random od pairs
% figure a: x = topologies, y = # of congested links, lines = monplan, escala, int-path, OPT (Gurobi)
% figure b: x = topologies, y = loss rate, lines = monplan, escala, int-path, OPT (Gurobi)
% figure c: x = topologies, y = time, lines = monplan, escala, int-path, OPT (Gurobi)


% % %\input{related-work}
\section{Conclusion}

We propose \sysname, a framework that co-designs sketches and INT to measure large and small flows respectively for accurate and resource-efficient network measurement. To enable such a co-design, 
\sysname solves the NP-hard measurement point selection with the near-optimal Lagrangian relaxation. It offers congestion-free data collection with worst-case rate estimations of sending measurement data and RL-based dynamic path selection. We implement
\sysname on 64$\times$400\,Gbps Tofino switches. Our testbed experiments indicate that \sysname offers remarkable accuracy improvement (e.g., 5.1$\times$ higher F1 scores) while achieving low resource consumption.




\clearpage
%\balance

%{\small
%\bibliographystyle{abbrv}
%\bibliography{paper}
%}

%\section*{Acknowledgement}

%We thank our reviewers for their insightful and constructive comments. 
%\para{Acknowledgement}. We thank reviewers for their comments. This work is supported by ``Pioneer'' and ``Leading Goose'' R\&D Program of Zhejiang (2024C01066), Quan Cheng Laboratory (QCLZD202304), Natural Science Foundation of China (61902362, 62172007), the Provincial Key R\&D Program of Zhejiang (2021C01032), the Yangzhou Science and Technology Plan Project (YZ2023200), Self-Developing Experimental Instrument and Equipment Project of Yangzhou University (zzyq2023zy06), the Fundamental Research Funds for the Central Universities (Zhejiang University NGICS Platform), and the Major Science and Technology Infrastructure Project of Zhejiang Lab (Large-scale experimental device for information security of new generation industrial control system).

%This work is supported by the National Key R\&D Program of China (2019YFB1802600), the National Natural Science Foundation of China (61902362, and 62172007), the Joint Funds of the National Natural Science Foundation of China (U20A20179), the Fundamental Research Funds for the Central Universities (Zhejiang University NGICS Platform), and the Major Science and Technology Infrastructure Project of Zhijiang Laboratory (Large-scale experimental device for information security of new generation industrial control system).

%the Key R\&D Program of Zhejiang Province (2021C01036), 

%\clearpage
\renewcommand\refname{Reference}
\bibliographystyle{IEEEtran}
{\footnotesize
\bibliography{paper}}

% that's all folks
\end{document}
