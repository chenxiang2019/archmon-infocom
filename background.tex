\section{Background and Motivation}

\subsection{Our Scope: Network Measurement and Applications}

We first illustrate our scope in this paper. We follow existing studies \cite{namkung2022sketchlib,anup2022hetero,liu2016one} to consider a general architecture of network measurement comprising two planes. In the data plane, switches execute sketches and INT techniques to collect traffic statistics to a cluster of servers that collectively act as the control plane. The control plane runs network management applications that query statistics to identify network events and make decisions. We define relevant terms without loss of generality as follows.

\begin{itemize}[leftmargin=*]
%
    \item Each packet can be viewed as a tuple of header fields (e.g., IP) and metadata (e.g., packet size and location traversed).
%
    \item Each flow groups the packets that share the same set of header fields and metadata, e.g., the same source IP address. 
%
    \item Traffic statistics, or per-flow statistics, correspond to the flow attributes of interest, e.g., per-flow packet counts. 
%
\end{itemize}

\noindent With the above terms, we classify network management applications into three classes based on their differences on queries.

\begin{itemize}[leftmargin=*]
%
    \item Volumetric applications, including heavy hitter detection \cite{huang2017sketchvisor}, superspreader detection \cite{tang2019mv}, and DDoS flow detection \cite{liu2021jaqen}, query the size of each flow (e.g., packet or byte counts). 
%
    \item Aggregated applications, e.g., traffic cardinality and entropy estimation \cite{liu2016one}, aggregate statistics to summarize all flows. 
%
    \item Troubleshooting applications, e.g., path and latency monitoring \cite{ben2020pint,sheng2021deltaint}, query per-flow metadata (e.g., timestamps).
%
\end{itemize}

\noindent Also, since measurements may suffer from errors, the literature introduces several application-level accuracy metrics to evaluate the quality of measurements: the precision, \emph{tp}/(\emph{tp}+\emph{fp}), the recall, \emph{tp}/(\emph{tp}+\emph{fn}), and the F1 score, 2\emph{tp}/(2\emph{tp}+\emph{fp}+\emph{fn}), where \emph{tp}, \emph{fp}, and \emph{fn} denote the number of true positives, false positives, and false negatives, respectively. 

\subsection{Preliminaries on Sketches}

Sketches are the approximate algorithms that provide near-optimal accuracy guarantees to the measurement of large flows. They keep their resource consumption below a limited resource budget. So they are compatible with resource-constrained data plane switches. Prior studies \cite{li2016flowradar,yang2018elastic,huang2017sketchvisor,huang2018sketchlearn,liu2016one,huang2021toward,liu2019nitrosketch,zhang2021cocosketch,namkung2022sketchlib} have shown that sketches provide better accuracy-resource tradeoffs for large flows when compared with traditional sampling-based techniques such as NetFlow \cite{netflow}. Note that our work is not to propose new sketches. Instead, we focus on leveraging existing sketches to enable better measurements. 

\para{Limitation 1 (L1):} \emph{Sketches can hardly provide high accuracy for small flows}. Sketches can only use a few memory in their data structures due to switch resource constraints. As a result, the data of large flows and small flows easily collide, resulting in measurement errors. In particular, the measurement of small flows is extremely sensitive to such errors because these flows have very few packets and may be missed entirely or their small counts may be overestimated due to hash collisions. 

We validate our claim with testbed experiments. We consider five sketches and employ them for the application of per-flow packet counting, i.e., the count-min sketch (CM) \cite{cormode2005s}, the count sketch (CS) \cite{charikar2004finding}, the elastic sketch (ES) \cite{yang2018elastic}, SketchLearn (SL) \cite{huang2018sketchlearn}, and UnivMon (UM) \cite{liu2016one}. We consider the de-facto standard traffic traces, CAIDA \cite{caida}, and partition them into two-second intervals, each containing 100\,K flows. Then for each sketch, we allocate 10\,MB to its data structures because this configuration approximates the maximum memory capacity per switch \cite{gupta2018sonata}. In Figure~1, we present the ratio of flows, which measurement errors are below 10\%, in each sketch. All ratios are lower than 50\%. The reason is that most small flows suffer from non-trivial errors due to memory shortage. If we allocate sufficient memory to each sketch (i.e., ``Ideal''), the accuracy would be improved. However, this is infeasible in practice due to switch constraints. 

Some recent studies have been proposed to tackle the above issue. For example, NZE-sketch \cite{huang2021toward} leverages the theories of compressive sensing to design sketches to approximately record traffic data in switches, and recover original data in the control plane. Moreover, learning-based sketches such as TalentSketch \cite{li2024learning} leverage machine learning models to identify which flows suffer from low measurement accuracy and update sketch data structures in subsequent measurements. However, they exhibit similar issues. First, they need specific sketch design to enforce their optimizations, losing the generality of supporting different types of sketches or network management applications. Second, their recovery or updating strategies require O(10$^2$) seconds, making them unsuitable for a lot of scenarios in which measurements should be made within a few milliseconds. For example, DDoS detection addresses sub-millisecond detection time \cite{liu2021jaqen}. 

\subsection{Preliminaries on In-Band Network Telemetry (INT)}

The core idea of INT is simple yet effective. For each packet, INT orders every switch that the packet traverses to piggyback the information required for measurement onto the packet's INT header. At the egress switch, where packets leave the network, their INT headers are extracted and emitted to the control plane for further analysis. Thus, INT preserves full accuracy for every flow, especially for small flows. 

\para{Limitation 2 (L2):} \emph{INT can bring significant network resource overheads}. Since INT corresponds to per-packet monitoring, it inevitably brings significant overheads to both network bandwidth (for transferring INT headers) and control plane resources (for processing INT headers) given massive packet number. In theory, according to the INT protocol \cite{int}, each switch will add a 12-byte INT header to each packet while the total number of INT headers equals the number of hops traversed by the packet. Given that modern networks transfer Tbps-level traffic, the total number of packets per second is O(10$^9$), corresponding to the number of INT headers to be emitted simultaneously. Hence, although 12\,bytes seem to be small, the accumulation of O(10$^9$) INT headers becomes significant and unacceptable. 

Figure~1 highlights the impact of number of switches in the network on INT overheads. In detail, we consider the topology of a production network from one of the largest national Internet service providers. This network has 152 switches and 225 links. We vary the ratio of switches activating INT from 20\% to 100\%. We quantify the overheads in terms of number of INT headers to be collected with respect to real-world statistics (e.g., how many packets that a switch processes per second) collected from production. Our results show that even with 20\% switches, INT still introduces non-trivial overheads in terms of bandwidth and computational resources in the control plane. 

To date, many optimizations have been proposed to optimize INT overheads. First, some of them rely on sampling and only invoke INT for sampled packets \cite{tang2019sel,suh2020flexible,kim2018selective,ben2020pint}. Nevertheless, sampling hurts accuracy, especially for small flows with a few packets. Also, the control plane requires a long time for result convergence. Next, INT-path \cite{pan2019int} sends probes to determine the minimum set of non-overlapped paths that INT must measure. But its per-packet monitoring still suffers from high overheads. 
%
Second, some approaches propose to simplify INT information on each packet \cite{ben2020pint,zhao2021lightguardian,sheng2021deltaint}. PINT \cite{ben2020pint} reduces INT metadata via sampling-based global hashing and distributed encoding. But it inherits the issues of sampling. LightGuardian \cite{zhao2021lightguardian} uses small sketches to encode INT data in each packet in a compact manner. But it loses accuracy for small flows. 


\subsection{Opportunity of Combining Sketches and INT}

According to our analysis above, we can notice that sketches and INT complement each other. More precisely, in Table~I, we summarize the advantages and limitations of sketches and INT: Sketches exhibit both high accuracy and resource efficiency for large flows, but they fall short for small flows; INT can provide high accuracy for both large and small flows, but they come at the cost of high resource consumption. 

In this context, we conclude the opportunities of combining sketches and INT: we can use sketches to measure large flows while using INT to measure small flows. Doing so can achieve both high accuracy and resource efficiency. In detail, traffic in modern networks is skewed \cite{roy2015inside,huang2021toward,caida,benson2010network,yang2018elastic}. So most packets originate from large flows, which are accurately and efficiently measured by sketches. For other packets from small flows, they are measured by INT, ensuring their full accuracy. Also, their number is small, limiting INT's resource consumption. 

\para{Challenges}.
However, while it is technically sound, leveraging these opportunities faces the following three challenges.

\begin{itemize}[leftmargin=*]
%
    \item \textbf{C1: Measurement point selection with incomplete knowledge}. Traffic routing details are usually unknown in advance. Without such details, selecting measurement points (i.e., the switches that perform sketches and INT) is a non-trivial task due to the following reasons. First, the core difficulty lies in guaranteeing that every flow traverses at least one measurement point when routing paths are unknown. This uncertainty may result in over-provisioning of measurement points, increasing deployment overheads. Second, measurement points should be close to the control plane. Otherwise, the collection of measurement data may suffer from high latency, delaying subsequent network management. Thus, without precise path knowledge, achieving both high coverage and low collection latency becomes high-dimensional NP-hard optimizations. 
%
    \item \textbf{C2: General data reduction for heterogeneous measurement data}. For resource efficiency, measurement data should be minimized to save network bandwidth and control plane resources. However, such reduction faces the inherent heterogeneity between sketch data (e.g., structured counter arrays) and INT headers (e.g., unstructured hop-by-hop metadata). Specifically, reducing sketch data needs to preserve counter relationships over time, while reducing INT data should avoid modifying original per-switch information. Such differences impede the design of reducing measurement data. Without careful analysis, simply collaborating existing data reduction techniques would be ineffective and even hurt data integrity. 
%
    \item \textbf{C3: Congestion-free measurement data collection under bursts}. Measurement data from sketches and INT generates data streams that may saturate network paths connecting measurement points with the control plane. Thus, this challenge stems from the need to prevent the streams of measurement data from causing congestion during collection. For example, when thousands of incoming flows simultaneously activate INT at the switch, the resulting O(10$^2$)\,Gbps data traffic can overwhelm 10-Gbps paths in milliseconds. Profiling ``safe'' collection rates alone is insufficient because traffic dynamics cause rates to vary unpredictably. Thus, avoiding congestion requires complicated modeling of worst-case collection rates and reserving bandwidth without starving normal traffic.
%
\end{itemize}

%To maximize resource efficiency, measurement data should be aggressively reduced. But existing techniques \cite{sheng2021deltaint,chen2021mtp,liu2022escala} only consider one type of measurement data and naturally fail to handle the fundamental heterogeneity between sketch data (i.e., sketch counter arrays) and INT data (i.e., INT headers). Also, reducing data must preserve integrity, i.e., 

%For example, many approaches \cite{sheng2021deltaint,chen2021mtp,liu2022escala} only consider one type of measurement data. Thus, they naturally lack a uniform way to reduce heterogeneous data. 

%Traditional static data reduction techniques fail to handle skewed and time-varying traffic patterns, leading to either insufficient compression (wasting bandwidth) or over-aggregation (missing critical details). This is because measurement data scale depends on the flow size: large flows tolerate summarization, while small flows require precision. For example, sketch data alone can be compressed by merging similar values, but INT data may reduce a 1MB sketch update to 100KB during stable periods but fails during traffic shifts (e.g., DDoS onset), where 95\% of counters change simultaneously. Similarly, caching frequent items risks missing ephemeral but critical microburst indicators. This forces inefficient tradeoffs: aggressive reduction distorts small-flow accuracy, while conservative approaches flood control planes during volatility.

%TowerSketch \cite{}

