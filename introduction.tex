\section{Introduction}

Network measurement has been a fundamental component in modern networks. For example, it enables traffic monitoring and engineering in data center networks \cite{li2016flowradar,woodruff2019measuring,zhang2017high} and wide-area networks \cite{huang2017sketchvisor,liu2016one,caida}, and supports network latency monitoring in distributed large language model training scenarios \cite{qian2024alibaba}. More precisely, it measures real-time traffic statistics such as per-flow packet counts in data plane switches and periodically collects these measurement data to the control plane. The control plane offers data to network management applications for identifying the events of their interest and making corresponding decisions. 

However, existing measurement techniques face the dilemma of achieving high accuracy and resource efficiency simultaneously. More precisely, the literature has proposed two categories of measurement techniques, i.e., sketches and in-band network telemetry (INT). For sketches \cite{li2016flowradar,yang2018elastic,huang2017sketchvisor,huang2018sketchlearn,liu2016one,huang2021toward,liu2019nitrosketch,zhang2021cocosketch,namkung2022sketchlib}, they achieve accurate measurement for large flows via compact data structures. While sketches are resource-efficient, they exhibit low accuracy when measuring small flows due to the short of switch memory. For INT-based approaches \cite{sheng2021deltaint,ben2020pint,handigol2014know,pan2019int,zhu2015packet}, they record statistics within the INT headers of each packet and collect these headers at the egress of networks. Although their per-packet monitoring achieves full accuracy, it imposes high pressure on network bandwidth and control plane resources to handle large volumes of INT data. 

Although recent studies attempt to break the above dilemma, they come at additional cost. Some of them propose sophisticated sketches such as elastic sketches \cite{yang2018elastic} and pyramid sketches \cite{yang2017pyramid} that distinguish the measurement of small flows from large flows to mitigate the errors originated from hashing collisions. However, they involve complex data structures and operations, which hinder their hardware implementation, and also require time-consuming parameter tuning. Also, a lot of optimizations aim at reducing INT overheads via strategies like path planning \cite{pan2019int} or probabilistic updating \cite{ben2020pint}. Nevertheless, given the large volume of large flows, massive packets collectively force INT to generate a high-speed stream of INT headers, which brings non-trivial pressure to the control plane. 

Hereby, we issue the following question: how can we achieve accurate and resource-efficient measurement towards both large and small flows? Our answer is \sysname, which is an accurate and resource-efficient measurement framework. Our core idea originates from our observation: sketches and INT techniques are \emph{complementary} to each other. More precisely, sketches trade their accuracy of measuring small flows for resource efficiency at scale, while INT techniques trade their resource efficiency for full accuracy of measuring every flow. Hence, their advantages and limitations lead to the possible match: for traffic in modern networks, which is considered skewed (i.e., most packets come from a few large flows) \cite{roy2015inside,huang2021toward,caida,benson2010network,yang2018elastic}, \sysname leverages sketches to measure large flows while using INT techniques to measure small flows, maximizing both accuracy and efficiency.

While this concept is simple, combining sketches with INT in practice faces two optimization challenges: where to install sketches and INT to measure as many flows as possible when routing information is unknown in advance, and how to prevent measurement data collection from congesting network paths and affecting normal traffic remain hard optimization problems. 
To address these challenges, \sysname offers two optimizations, including measurement point selection that yields near-optimal decisions under unknown routing information, and congestion-free measurement data collection that profiles worst-case estimates to determine loss-free decisions. 

%(2) resource-efficient measurement data reduction that efficiently and safely reduces diverse types of data, 

%how to efficiently reduce the large amount of measurement data to improve resource efficiency while keeping data integrity, 

\para{Contributions}. 
%However, co-designing sketches and INT techniques within the same system faces the fundamental challenges of \emph{where}, \emph{how}, and \emph{how efficiently} to deploy these techniques. 
%we design \sysname to bridge the accuracy-efficiency gap by systematically integrating sketches and INT. 
We summarize our contributions as follows.  

\begin{itemize}[leftmargin=*]
%
    \item We present an in-depth analysis towards the advantages and limitations of existing measurement techniques. Our analysis presents that sketches provide accurate and resource-efficient measurements for large flows but fall short for small flows, while INT accurately monitors all flows but suffers from high resource overheads (\S\ref{background}). 
%
    \item According to analysis results, combining sketches and INT presents an opportunity of realizing both high accuracy and resource efficiency simultaneously. We exploit this opportunity by proposing \sysname, a framework that uses sketches to measure large flows while invoking INT for small flows, enabling accurate and resource-efficient measurements (\S\ref{overview}). 
%
    \item However, co-designing sketches and INT faces the challenges of \emph{where} to deploy them and \emph{how} to collect measurement data. \sysname provides two optimizations, i.e., near-optimal measurement point selection, and congestion-free measurement data collection to handle these challenges (\S\ref{selection}-\S\ref{collection}). 
    % resource-efficient measurement data reduction, as well as
%
    \item We implement \sysname on our testbed that comprises $64\times 400$\,Gbps Tofino2 switches \cite{tofino2}. Testbed experiments indicate that \sysname outperforms existing standalone sketches or INT techniques with accuracy improvement (e.g., $3\times$ F1 scores) and orders-of-magnitude lower overheads. 
%
\end{itemize}

