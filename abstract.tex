\begin{abstract}
Network measurement is now a fundamental building block that enables network management to detect and handle real-time events (e.g., rejecting malicious flows). However, state-of-the-art techniques present a trade-off between high accuracy and resource efficiency: sketches enjoy low resource consumption when measuring large flows. But they sacrifice accuracy for small flows. Also, in-band network telemetry (INT) accurately measures each flow at the cost of significant resources.
In this paper, we present \sysname, a framework that utilizes sketches and INT to keep track of large and small flows, respectively, simultaneously achieving high accuracy and resource efficiency. More precisely, unlike existing studies, our co-design is driven by two theoretical optimizations: (1) \sysname adopts the near-optimal Lagrangian relaxation to solve the NP-hard selection of measurement points to deploy sketches and INT with unknown routing information. (2) \sysname estimates the worst-case rates of collecting measurement data. Its estimates support reinforcement learning to select paths without congestions.  
Experiments on 64$\times$400\,Gbps Tofino switches indicate that compared to existing techniques, \sysname provides higher accuracy (e.g., $5\times$ F1 scores) and lower resource consumption (e.g., reducing bandwidth overhead by 70\%). 
\end{abstract}

%Sketches naturally produce accurate results for large flows with resource efficiency. Next, \sysname uses INT to transfer the data of small flows to the control plane. As small flows only have a few packets, it limits INT's resource consumption. 
