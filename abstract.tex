\begin{abstract}
Network measurement is now a fundamental building block that enables network management to detect and handle real-time events (e.g., rejecting malicious flows). However, state-of-the-art techniques present a trade-off between high accuracy and resource efficiency: sketches enjoy low resource consumption when measuring large flows. But they sacrifice accuracy for small flows. In-band network telemetry (INT) supports fully accurate measurements for each flow but requires significant resources.

In this paper, we present \sysname, a framework that improves measurement accuracy without sacrificing resource efficiency. Its key idea is to co-design sketches and INT to collectively achieve high accuracy and resource efficiency for arbitrary flows. More precisely, it selects sufficient network locations to place sketches with the goal of covering as many large flows as possible. Sketches naturally produce accurate results for large flows with resource efficiency. Next, \sysname uses INT to transfer the data of small flows to the control plane. As small flows only have a few packets, it limits INT's resource consumption. Evaluation on 64$\times$400\,Gbps Tofino switches indicates that compared to standalone techniques, \sysname offers remarkable accuracy improvement to small flows (e.g., $5\times$ F1 scores) while achieving low resource consumption. 
\end{abstract}
