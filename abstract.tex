\begin{abstract}
Network measurement is now a fundamental building block that enables network management to detect and handle real-time events (e.g., rejecting malicious flows). However, state-of-the-art techniques present a trade-off between high accuracy and resource efficiency: sketches enjoy low resource consumption when measuring large flows. But they sacrifice accuracy for small flows. In-band network telemetry (INT) supports fully accurate measurements for each flow but necessitates significant resources.
In this paper, we present \sysname, a framework that leverages sketches and INT to measure large and small flows, respectively, simultaneously achieving high accuracy and resource efficiency for arbitrary flows. Such co-design is supported by two-pronged optimizations. First, \sysname exploits the Lagrangian relaxation to solve the NP-hard problem of selecting sufficient switches to deploy sketches and INT even with unknown routing information. Second, it estimates the worst-case rates of collecting sketch data and INT data. According to its estimates, it picks enough paths to transfer data without congestion.  
Evaluation on 64$\times$400\,Gbps Tofino switches indicates that compared to standalone techniques, \sysname offers remarkable accuracy improvement to small flows (e.g., $5\times$ F1 scores) while achieving low resource consumption. 
\end{abstract}

%Sketches naturally produce accurate results for large flows with resource efficiency. Next, \sysname uses INT to transfer the data of small flows to the control plane. As small flows only have a few packets, it limits INT's resource consumption. 
